\documentclass[14pt, hyperref = {colorlinks}]{beamer}
\usepackage[T2A]{fontenc}
\usepackage[utf8]{inputenc}
\usepackage[english,russian]{babel}
\usepackage[colorlinks]{hyperref}
\usepackage{amssymb,amsfonts,amsmath,mathtext}
\usepackage{cite,enumerate,float,indentfirst}

%Диаграммы Фейнмана
\usepackage{tikz}
\usetikzlibrary{trees}
\usetikzlibrary{decorations.pathmorphing}
\usetikzlibrary{decorations.markings}
\usetikzlibrary{positioning, automata, arrows, arrows.meta}

\graphicspath{{images/}}

\usetheme{Pittsburgh}
\usecolortheme{whale}

\setbeamercolor{footline}{fg=blue}
\setbeamertemplate{footline}{
  \leavevmode%
  \hbox{%
  \begin{beamercolorbox}[wd=.333333\paperwidth,ht=2.25ex,dp=1ex,center]{}%
    A.Semenov
  \end{beamercolorbox}%
  \begin{beamercolorbox}[wd=.333333\paperwidth,ht=2.25ex,dp=1ex,center]{}%
    Novosibirsk, 2017
  \end{beamercolorbox}%
  \begin{beamercolorbox}[wd=.333333\paperwidth,ht=2.25ex,dp=1ex,right]{}%
  Page \insertframenumber{} / \inserttotalframenumber \hspace*{2ex}
  \end{beamercolorbox}}%
  \vskip0pt%
}

\newcommand{\itemi}{\item[\checkmark]}

\title{\small{Study the $e^{+}e^{-} \to K_{S}K_{L}\pi^{0}$ process with the CMD-3 detector}}
\author{\small{%
\emph{Reporter:}~A.Semenov\\%
\emph{Supervisor:}~PhD.~B.Shwartz}\\%
\vspace{30pt}%
BINP
\vspace{0pt}%
}

\date{\includegraphics[width=0.2\linewidth]{logo} \\
\vspace{20pt}% 
\small{Novosibirsk, 2017}}

\begin{document}

\maketitle
\small
\begin{frame}
\frametitle{Outline}\label{t1}
\begin{itemize}
  \item Objective and motivation
  \item VEPP-2000 project
  \item List of the selection constraints
  \item Results
  \item Conclusion and future plans
\end{itemize}
\end{frame}

\begin{frame}
\frametitle{Objective and motivation}\label{t1}
The main goal of the current work is to measure the $e^{+}e^{-} \to K_{S}K_{L}\pi^{0}$ process cross-section up to 2 GeV in the mass center system.
\begin{itemize}
  \item Study the light quarks interaction
  \item Non-perturbative QCD
  \item Contribution of this process into the processes with two $K$ and one $\pi$ equals 12\%.
  \item Anomalous magnetic moment {$(g-2)_\mu$}.
\end{itemize}
\end{frame}

\begin{frame}
\frametitle{VEPP-2000 project}
\begin{figure}[h]
\begin{minipage}[h]{0.49\linewidth}   
    \center{\includegraphics[width=1.1\linewidth]{vepp2000}}
\end{minipage}
\begin{minipage}[h]{0.49\linewidth}
    \center{\includegraphics[width=1.1\linewidth]{cmd}}
\end{minipage}
\center{Data of 2011-2012 seasons\\Integrated luminosity is 33.18 pb$^{-1}$}
\hfill
\end{figure}
\end{frame}

\begin{frame}\label{r}
\frametitle{Preliminary event selection}
\begin{figure}[h]
\begin{minipage}[h]{0.59\linewidth}

\center{The next decay modes are used:}
\begin{itemize}
    \item $K_{S} \to \pi^{+} \pi^{-}$ (69\%)
    \item $\pi_{0} \to \gamma \gamma$ (99\%)
\end{itemize}
\center{Preliminary selection cuts:}
\begin{itemize}
  \item {Number of tracks, $N_{tr} = 2$} 
  \item {Number of photons, $N_{ph} \geq 2$}
  \item {One vertex of a $K_{S}$ meson, $N_{K_{S}} = 1$}
\end{itemize}
\end{minipage}
\begin{minipage}[h]{0.39\linewidth}
    \center{
        \center{\subcaptionbox{$e^{+}\:e^{-} \to K^{*}\:\bar{K}$}}
        {\tikzset{
    photon/.style={decorate, decoration={snake}, draw=black},
    electron/.style={draw=black, postaction={decorate},
        decoration={markings,mark=at position .55 with {\arrow[draw=black]{<}}}},
    positron/.style={draw=black, postaction={decorate},
        decoration={markings,mark=at position .55 with {\arrow[draw=black]{>}}}},
    scalar/.style={decorate, dashed, draw=black} ,
    vector/.style={decorate, double, double distance=3pt, draw=black} ,
    line/.style={decorate, draw=black} ,
    gluon/.style={decorate, draw=magenta,
        decoration={coil,amplitude=4pt, segment length=5pt}} 
}

\resizebox{4cm}{!}{
\begin{tikzpicture}[
        thick,
        % Set the overall layout of the tree
        level/.style={level distance=1.5cm},
        level 2/.style={sibling distance=2.6cm},
        level 3/.style={sibling distance=2cm}
    ]
    \coordinate
        child[grow=left]{
            child {
                node {$e^{-}$}
                % The 'edge from parent' is actually not needed because it is
                % implicitly added.
                edge from parent [electron]
            }
            child {
                node {$e^{+}$}
                edge from parent [positron]
            }
            edge from parent [photon] node [above=3pt] {$\gamma$}  
        }
        % I have to insert a dummy child to get the tree to grow
        % correctly to the right.
        child[grow=right, level distance=0pt] {
        child  {
            child{
                edge from parent [scalar] node [below=3pt] {$\bar{K}$}
            }
            child{
                child{
                    edge from parent [scalar] node [below=3pt] {$K$}
                }
                child{
                    edge from parent [scalar] node [above=3pt] {$\pi^{0}$}
                }
                edge from parent [line] node [above=3pt] {$K^{*}$}
            }
            edge from parent [vector] node [above=3pt] {$V$}     
        }
    };
\end{tikzpicture}
}} 
        \center{\subcaptionbox{$e^{+}\:e^{-} \to \phi\:\pi^{0}$}}
        {\input{Diagrams/Phi}}
    }
\end{minipage}
\end{figure}
\end{frame}

\begin{frame}\label{r}
\frametitle{Event selection}
\begin{itemize}
  \item {Ionization losses {$({dE\over{dx}})_{\pi}$}}
  \item {The cosine of the angle between the momentum and the position vector of the $K_{S}$ vertex in the XY plane is more than 0.8}
  \item {The solid angle between tracks is more than theoretical mean {$\psi > \psi_{min}$}}
  \item {The energy release of the photon signal in the LXe or in BGO-calorimeters is more than threshold energy}
  \item {The solid angle between signal photons is more than theoretical mean}
  \item {The polar angles of tracks are in the range ($0.9, \pi-0.9$)}
\end{itemize}
\end{frame}

\begin{frame}\label{r3}
\frametitle{Ionization losses}
\begin{figure}[h]
\center\textbf{The beams energy is 840 MeV}
\begin{minipage}[h]{0.69\linewidth}
        \center{\includegraphics[width=0.8\linewidth]{Present/images/tdedx.png}}
\end{minipage}
\hfill
\begin{minipage}[h]{0.29\linewidth}
    \small{$({dE\over{dx}})_{\pi}(p) = a + {b\over{p^{2}}}$
    \\$a = 2047.2\\b = 2.3*10^{7}$}
    \begin{itemize}
        \item {$N_{tr} = 2$}
        \item {$N_{ph} \geq 2$}
        \item {$N_{K_{S}} = 1$}
    \end{itemize}
\end{minipage}
\end{figure}
\end{frame}

\begin{frame}
\frametitle{Ionization losses}
\begin{figure}[h]
\center{$\xi_{tr} = {(dE/dx)_{exp}\over{(dE/dx)_{aprx}}}$}\\
\center{\textbf{The selection criterion is $\xi_{tr} < 1.6$}}\\
 \begin{minipage}[h]{0.49\linewidth}
    \center{Simulation\\(840 MeV, ISR on)}
    \center{\includegraphics[width=0.8\linewidth]{Present/images/xiSim.png}}
  \end{minipage}
  \hfill
  \begin{minipage}[h]{0.49\linewidth}
    \center{Experiment\\(840 MeV, ISR on)}
    \center{\includegraphics[width=0.8\linewidth]{Present/images/xiExp.png}}
  \end{minipage}
   \small\center{$N_{tr} = 2$, $N_{ph} \geq 2$, $N_{K_{S}} = 1$}
\end{figure}
\end{frame}

\begin{frame}
\frametitle{The cosine between the momentum and the radius-vector of the $K_{S}$}
\begin{figure}[h]
\center{\textbf{The selection criterion is $cos_{K_{S}} > 0.8$}}
\begin{minipage}[h]{0.69\linewidth}
    \center{\includegraphics[width=0.6\linewidth]{ksalign}}
    \center{\includegraphics[width=0.3\linewidth]{cos}}
\end{minipage}
\begin{minipage}[h]{0.29\linewidth}
    \begin{itemize}
        \item {$N_{tr} = 2$}
        \item {$N_{ph} \geq 2$}
        \item {$N_{K_{S}} = 1$}
        \item {$\xi_{tr} < 1.6$}
        \item {$E_{phlxe} > 15$ or $E_{phbgo} > 15$}
    \end{itemize}
\end{minipage}
\hfill
\end{figure}
\end{frame}

\begin{frame}\label{r1}
\frametitle{The minimal angle between the tracks}
\begin{figure}[h]
    \center\textbf{$\psi_{min} = 2\arctg{(\sqrt{M^{2} - 4m^{2}}/P)}$}\\
\begin{minipage}[h]{0.49\linewidth}   
    \center{\includegraphics[width=0.8\linewidth]{Present/images/ksdpsiSim.png}}
\end{minipage}
\begin{minipage}[h]{0.49\linewidth}
    \begin{itemize}
    \center{\includegraphics[width=1\linewidth]{Present/images/ksdpsiExp.png}}
    \end{itemize}
\end{minipage}
\center{
    {$N_{tr} = 2$, }
    {$N_{ph} \geq 2$, }
    {$N_{K_{S}} = 1$, }
    {$\xi < 1.6$, }
    {\small{$cos(\vec{r}_K_{s},\vec{P})>0.8$}, }
    {$E_{phlxe} > 15$ or $E_{phbgo} > 15$}
    }
\center{The condition of the pion's turn $\pi$, $P_{K_{S}} > 768$ MeV\\($E = 915$ MeV)}
\hfill
\end{figure}
\end{frame}

\begin{frame}\label{r1}
\frametitle{The minimal angle between the signal photons}
\begin{figure}[h]
    \center\textbf{$\alpha_{min} = 2\arctg{(m/P)}$}\\
\begin{minipage}[h]{0.49\linewidth}   
    \center{\includegraphics[width=0.8\linewidth]{Present/images/pi0dpsiSim.png}}
\end{minipage}
\begin{minipage}[h]{0.49\linewidth}
    \begin{itemize}
    \center{\includegraphics[width=1\linewidth]{Present/images/pi0dpsiExp.png}}
    \end{itemize}
\end{minipage}
\center{
    {$N_{tr} = 2$, }
    {$N_{ph} \geq 2$, }
    {$N_{K_{S}} = 1$, }
    {$\xi < 1.6$, }
    {\small{$cos(\vec{r}_K_{s},\vec{P})>0.8$}, }
    {$E_{phlxe} > 15$ or $E_{phbgo} > 15$}
    }
\hfill
\end{figure}
\end{frame}

\begin{frame}\label{r2}
\frametitle{Polar angles of the tracks}
\begin{figure}[h]
\center{\textbf{The selection criterion is $\theta_{tr} \in (0.9; \pi-0.9)$}}
\begin{minipage}[h]{0.69\linewidth}   
    \center{\includegraphics[width=0.8\linewidth]{Present/images/tthSim.png}}
\end{minipage}
\begin{minipage}[h]{0.29\linewidth}
    \begin{itemize}
        \item {$N_{tr} = 2$}
        \item {$N_{ph} \geq 2$}
        \item {$N_{K_{S}} = 1$}
        \item {$\xi_{tr} < 1.6$}
        \item {$E_{phlxe} > 15$ or $E_{phbgo} > 15$}
    \end{itemize}
\end{minipage}
\hfill
\end{figure}
\end{frame}

%\begin{frame}\label{r2}
%\frametitle{Results}
%\begin{figure}[h]
%  \begin{minipage}[h]{0.49\linewidth}
%    \center\textbf{Simulation \\(840 MeV, ISR on)}
%    \center{\includegraphics[width=0.8\linewidth]{Present/images/xiafterM.png%}}
%  \end{minipage}
%  \hfill
%  \begin{minipage}[h]{0.49\linewidth}
%    \center\textbf{Experiment\\(840 MeV)}
%    \center{\includegraphics[width=0.8\linewidth]{Present/images/xiafter.png}%}
%  \end{minipage}
%  \center{Dependence of the $\xi_{-}$ on the $\xi_{+}$}
%\end{figure}
%\end{frame}

\begin{frame}
\frametitle{Results\\(840 MeV)}
\begin{figure}[h]
  \begin{minipage}[h]{0.49\linewidth}
    \center\textbf{Simulation \\(840 MeV, ISR on)}
    \center{\includegraphics[width=0.8\linewidth]{Present/images/ks:piSim.png}}
  \end{minipage}
  \hfill
  \begin{minipage}[h]{0.49\linewidth}
    \center\textbf{Experiment \\(840 MeV)}
    \center{\includegraphics[width=0.8\linewidth]{Present/images/ks:piExp.png}}
  \end{minipage}
  \center{Dependence of the $K_{S}$ meson mass on the $\pi^{0}$ meson mass}
\end{figure}
\end{frame}

\begin{frame}
\frametitle{Effiency}
\center $\epsilon = {N_{det}\over{N}}$, where $N=10^{5}$.

2011 --- 1.0 T, 2012 --- 1.3 T
\begin{figure}[h]
    \center{\includegraphics[width=0.6\linewidth]{Present/images/effiency.png}}
\end{figure}
\end{frame}

\begin{frame}
\frametitle{Calculation of the cross section}
\begin{figure}[h]
\center{\includegraphics[width=0.5\linewidth]{Present/images/fitksminv.png}}
    \begin{itemize}
    \tiny{
    \item $S(x) = N*(\alpha*gausn(x, \bar{x}_{1}, \sigma_{1}) + \beta*gausn(x, \bar{x}_{2}, \sigma_{2}) + (1-\alpha-\beta)*gausn(x, \bar{x}_{3}, \sigma_{3})))$ --- signal
    \item $\Phi(x) = k*(x-b)$ --- background
    \item $S(x) + \Phi(x)$ --- total. 
    }
    \end{itemize}
\end{figure}
\end{frame}

\begin{frame}
\frametitle{Preliminary cross-section}
\center $\sigma = {N\over{\epsilon L}}$, where $N$ --- number of the good events, $\epsilon$ --- efficiency and $L$ --- integrated luminosity.\\
\begin{figure}[h]
    \center\textbf{}
    \center{\includegraphics[width=0.55\linewidth]{Present/images/cross.png}}
\end{figure}
\end{frame}

%\begin{frame}
%\frametitle{Background}
%Below 1400 MeV
%\begin{itemize}
%    \item $K_{S}K_{L}\gamma$
%    \item $\pi^{+}\pi^{-}(\gamma, \pi^{0}, 2\pi^{0})$
%\end{itemize}
%After 1400 MeV
%\begin{itemize}
%    \item $K_{S}K_{L}\pi^{0}\pi^{0}$
%    \item $K^{+}\pi^{-}\pi^{0}$
%    \item $\pi^{+}\pi^{-}(\gamma, \pi^{0}, 2\pi^{0})$
%\end{itemize}
%\end{frame}

\begin{frame}
\frametitle{Conclusion and future plans}
Conclusion:
\begin{itemize}
    \item The optimal criteria for selecting the events of the process have been developed.
    \item The efficiency of registration by means of Monte-Carlo simulation have been determined.
  \item The total cross-section of the process ${e^+\: e^ - \to K_{s}\:K_{L}\:\pi^0}$ in the energy range from \test{1.1 GeV} to \test{2 GeV} have been measured.
\end{itemize}
Future plans:
\begin{itemize}
    \item Detailed background study
    \item Publication
\end{itemize}
\end{frame}


\begin{frame}
\begin{center}
\textbf{Thank you for your attention!}
\end{center}
\end{frame}

\end{document} 