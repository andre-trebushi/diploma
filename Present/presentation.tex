\documentclass[14pt, hyperref = {colorlinks}]{beamer}
\usepackage[T2A]{fontenc}
\usepackage[utf8]{inputenc}
\usepackage[english,russian]{babel}
\usepackage[colorlinks]{hyperref}
\usepackage{amssymb,amsfonts,amsmath,mathtext}
\usepackage{cite,enumerate,float,indentfirst}

%Диаграммы Фейнмана
\usepackage{tikz}
\usetikzlibrary{trees}
\usetikzlibrary{decorations.pathmorphing}
\usetikzlibrary{decorations.markings}
\usetikzlibrary{positioning, automata, arrows, arrows.meta}

\graphicspath{{images/}}

\usetheme{Pittsburgh}
\usecolortheme{whale}

\setbeamercolor{footline}{fg=blue}
\setbeamertemplate{footline}{
  \leavevmode%
  \hbox{%
  \begin{beamercolorbox}[wd=.333333\paperwidth,ht=2.25ex,dp=1ex,center]{}%
    А. В. Семенов
  \end{beamercolorbox}%
  \begin{beamercolorbox}[wd=.333333\paperwidth,ht=2.25ex,dp=1ex,center]{}%
    Новосибирск, 2017
  \end{beamercolorbox}%
  \begin{beamercolorbox}[wd=.333333\paperwidth,ht=2.25ex,dp=1ex,right]{}%
  Стр. \insertframenumber{} из \inserttotalframenumber \hspace*{2ex}
  \end{beamercolorbox}}%
  \vskip0pt%
}

\newcommand{\itemi}{\item[\checkmark]}

\title{\small{Измерение сечения процесса e+ e- -> KS KL pi0 с детектором КМД-3}}
\author{\small{%
\emph{Выступающий:}~А.В.Семенов\\%
\emph{Руководитель:}~д.ф.-м.н.~Б. А. Шварц}\\%
\vspace{30pt}%
Институт Ядерной Физики
\vspace{0pt}%
}

\date{\includegraphics[width=0.2\linewidth]{logo} \\
\vspace{20pt}% 
\small{Новосибирск, 2017}}


\begin{document}

\maketitle

%\begin{frame}
%\frametitle{План}
%\begin{itemize}
%  \item Акутальность - \textbf{\ref{t1} стр.}
%  \item Цели - \textbf{\ref{t2} стр.}
%  \item Критерии отбора - \textbf{\ref{r} стр.}
%  \item {Результаты} - \textbf{\ref{r0} стр.}
%  \item {Планы} - \textbf{\ref{r5} стр.}
%\end{itemize}
%\end{frame}
\small
\begin{frame}
\frametitle{Цель работы}\label{t1}
\begin{center}
 Главной целью данной работы является измерение сечения процесса {$e^+\:e^- \to K_{S}\:K_{L}\:\pi^0$} в диапазоне энергий от порога рождения \text{(1130 МэВ)} до \text{2 ГэВ} в системе центра масс.
\begin{itemize}
  \item Изучение взаимодействия легких кварков.
  \item Конечные состояния с двумя каонами и пионом имеют 12\% вклад в адронное сечение при энергии 1650 МэВ.
  \item Вклад в поляризацию вакуума, используемое для вычисления аномального магнитного момента мюона {$(g-2)_\mu$}.
  \item Полученные сечения коллабораций SND и BaBar по данному процессу имеют различия в области выше 1800 МэВ в системе центра масс.
\end{itemize}
\end{center}
\end{frame}

%\begin{frame}
%\frametitle{Цели}\label{t2}
%\begin{itemize}
%  \item Измерить сечение процесса $e^{+}e^{-} \to KsKl\pi^{0}$ на детекторе КМД-3.
%\end{itemize}
%\end{frame}

\begin{frame}
\frametitle{КМД-3}
\begin{figure}[h]
 %   \center{\includegraphics[width=1.1\linewidth]{vepp2000}}
    \center{\includegraphics[width=0.8\linewidth]{cmd}}
\center{Использовались данные сезонов 2011 и 2012\\Интегральная светимость 33.18 пб$^{-1}$}
\hfill
\end{figure}
\end{frame}

\begin{frame}\label{r}
\frametitle{Предварительные критерии отбора}
\begin{figure}[h]
\begin{minipage}[h]{0.59\linewidth}

\center{Использовались следующие моды распадов:}
\begin{itemize}
    \item $K_{S} \to \pi^{+} \pi^{-}$ (69\%)
    \item $\pi_{0} \to \gamma \gamma$ (99\%)
\end{itemize}
\center{Предварительные критерии отбора:}
\begin{itemize}
  \item {Число треков $N_{tr} = 2$} 
  \item {Число фотонов $N_{ph} \geq 2$}
  \item {Одна вершина $N_{K_{S}} = 1$}
\end{itemize}
\end{minipage}
\begin{minipage}[h]{0.39\linewidth}
    \center{
        \center{\subcaptionbox{$e^{+}\:e^{-} \to K^{*}\:\bar{K}$}}
        {\tikzset{
    photon/.style={decorate, decoration={snake}, draw=black},
    electron/.style={draw=black, postaction={decorate},
        decoration={markings,mark=at position .55 with {\arrow[draw=black]{<}}}},
    positron/.style={draw=black, postaction={decorate},
        decoration={markings,mark=at position .55 with {\arrow[draw=black]{>}}}},
    scalar/.style={decorate, dashed, draw=black} ,
    vector/.style={decorate, double, double distance=3pt, draw=black} ,
    line/.style={decorate, draw=black} ,
    gluon/.style={decorate, draw=magenta,
        decoration={coil,amplitude=4pt, segment length=5pt}} 
}

\resizebox{4cm}{!}{
\begin{tikzpicture}[
        thick,
        % Set the overall layout of the tree
        level/.style={level distance=1.5cm},
        level 2/.style={sibling distance=2.6cm},
        level 3/.style={sibling distance=2cm}
    ]
    \coordinate
        child[grow=left]{
            child {
                node {$e^{-}$}
                % The 'edge from parent' is actually not needed because it is
                % implicitly added.
                edge from parent [electron]
            }
            child {
                node {$e^{+}$}
                edge from parent [positron]
            }
            edge from parent [photon] node [above=3pt] {$\gamma$}  
        }
        % I have to insert a dummy child to get the tree to grow
        % correctly to the right.
        child[grow=right, level distance=0pt] {
        child  {
            child{
                edge from parent [scalar] node [below=3pt] {$\bar{K}$}
            }
            child{
                child{
                    edge from parent [scalar] node [below=3pt] {$K$}
                }
                child{
                    edge from parent [scalar] node [above=3pt] {$\pi^{0}$}
                }
                edge from parent [line] node [above=3pt] {$K^{*}$}
            }
            edge from parent [vector] node [above=3pt] {$V$}     
        }
    };
\end{tikzpicture}
}} 
        \center{\subcaptionbox{$e^{+}\:e^{-} \to \phi\:\pi^{0}$}}
        {\input{Diagrams/Phi}}
    }
\end{minipage}
\end{figure}
\end{frame} 

\begin{frame}\label{r}
\frametitle{Критерии отбора}
\begin{itemize}
  \item {Отбор по ионизационным потерям треков {$({dE\over{dx}})_{\pi}$}} %  - \textbf{\ref{r3} стр.}
  \item {Косинус угла между импульсом $K_{S}$-мезона и радиус-вектором его вершины в плоскости XY больше, чем {$0.8$}}
  \item {Полярный угол треков лежит в диапазоне от $0.9$ радиан до $\pi-0.9$ радиан}
  \item {Пространственный угол между треками больше теоретического минимума {$\psi > \psi_{min}$}}
  \item {Энерговыделение фотона в LXe или в BGO-калориметрах больше порогового значения {$E_{phlxe} > 15$ МэВ || $E_{phbgo} > 15$ МэВ}}
  \item {Угол между фотонами больше теоретического минимума {$\alpha > \alpha_{min}$}}
\end{itemize}
\end{frame}

\begin{frame}\label{r3}
\frametitle{Ионизационные потери}
\begin{figure}[h]
\center\textbf{Ионизационные потери заряженных пионов при энергии 840 МэВ в пучке}
\begin{minipage}[h]{0.69\linewidth}
        \center{\includegraphics[width=0.8\linewidth]{Present/images/tdedx.png}}
\end{minipage}
\hfill
\begin{minipage}[h]{0.29\linewidth}
    \small{$({dE\over{dx}})_{\pi}(p) = a + {b\over{p^{2}}}$
    \\$a = 2047.2\\b = 2.3*10^{7}$}
    \begin{itemize}
        \item {$N_{tr} = 2$}
        \item {$N_{ph} \geq 2$}
        \item {$N_{K_{S}} = 1$}
    \end{itemize}
\end{minipage}
\end{figure}
\end{frame}

\begin{frame}
\frametitle{Ионизационные потери}
\begin{figure}[h]
\center{$\xi_{tr} = {(dE/dx)_{exp}\over{(dE/dx)_{aprx}}}$}\\
\center{\textbf{Критерий отбора $\xi_{tr} < 1.6$}}\\
 \begin{minipage}[h]{0.49\linewidth}
    \center{Моделирование\\(840 МэВ, ISR on)}
    \center{\includegraphics[width=0.8\linewidth]{Present/images/xiSim.png}}
  \end{minipage}
  \hfill
  \begin{minipage}[h]{0.49\linewidth}
    \center{Эксперимент\\(840 МэВ, ISR on)}
    \center{\includegraphics[width=0.8\linewidth]{Present/images/xiExp.png}}
  \end{minipage}
   \small\center{$N_{tr} = 2$, $N_{ph} \geq 2$, $N_{K_{S}} = 1$}
\end{figure}
\end{frame}

\begin{frame}
\frametitle{Косинус угла между вектором импульса и радиус-вектором вершины Ks-мезона}
\begin{figure}[h]
\center{\textbf{Критерий отбора $cos_{K_{S}} > 0.8$}}
\begin{minipage}[h]{0.69\linewidth}
    \center{\includegraphics[width=0.6\linewidth]{ksalign}}
    \center{\includegraphics[width=0.3\linewidth]{cos}}
\end{minipage}
\begin{minipage}[h]{0.29\linewidth}
    \begin{itemize}
        \item {$N_{tr} = 2$}
        \item {$N_{ph} \geq 2$}
        \item {$N_{K_{S}} = 1$}
        \item {$\xi_{tr} < 1.6$}
        \item {$E_{phlxe} > 15$ или $E_{phbgo} > 15$}
    \end{itemize}
\end{minipage}
\hfill
\end{figure}
\end{frame}

\begin{frame}\label{r1}
\frametitle{Минимальный угол между треками}
\begin{figure}[h]
    \center\textbf{$\psi_{min} = 2\arctg{(\sqrt{M^{2} - 4m^{2}}/P)}$}\\
\begin{minipage}[h]{0.49\linewidth}   
    \center{\includegraphics[width=0.8\linewidth]{Present/images/ksdpsiSim.png}}
\end{minipage}
\begin{minipage}[h]{0.49\linewidth}
    \begin{itemize}
    \center{\includegraphics[width=1\linewidth]{Present/images/ksdpsiExp.png}}
    \end{itemize}
\end{minipage}
\center{Условие разворота $\pi$, $P_{K_{S}} > 768$ МэВ\\($E = 915$ МэВ)}
\center{
    {$N_{tr} = 2$, }
    {$N_{ph} \geq 2$, }
    {$N_{K_{S}} = 1$, }
    {$\xi < 1.6$, }
    {\small{$cos(\vec{r}_K_{s},\vec{P})>0.8$}, }
    {$E_{phlxe} > 15$ или $E_{phbgo} > 15$}
    }
\hfill
\end{figure}
\end{frame}

\begin{frame}\label{r1}
\frametitle{Минимальный угол между фотонами}
\begin{figure}[h]
    \center\textbf{$\alpha_{min} = 2\arctg{(m/P)}$}\\
\begin{minipage}[h]{0.49\linewidth}   
    \center{\includegraphics[width=0.8\linewidth]{Present/images/pi0dpsiSim.png}}
\end{minipage}
\begin{minipage}[h]{0.49\linewidth}
    \begin{itemize}
    \center{\includegraphics[width=1\linewidth]{Present/images/pi0dpsiExp.png}}
    \end{itemize}
\end{minipage}
\center{
    {$N_{tr} = 2$, }
    {$N_{ph} \geq 2$, }
    {$N_{K_{S}} = 1$, }
    {$\xi < 1.6$, }
    {\small{$cos(\vec{r}_K_{s},\vec{P})>0.8$}, }
    {$E_{phlxe} > 15$ или $E_{phbgo} > 15$}
    }
\hfill
\end{figure}
\end{frame}

\begin{frame}
\frametitle{Полярный угол треков}
\begin{figure}[h]
\center{\textbf{Критерий отбора $\theta_{tr} \in (0.9; \pi-0.9)$}}
\begin{minipage}[h]{0.69\linewidth}   
    \center{\includegraphics[width=0.8\linewidth]{Present/images/tthSim.png}}
\end{minipage}
\begin{minipage}[h]{0.29\linewidth}
    \begin{itemize}
        \item {$N_{tr} = 2$}
        \item {$N_{ph} \geq 2$}
        \item {$N_{K_{S}} = 1$}
        \item {$\xi_{tr} < 1.6$}
        \item {$E_{phlxe} > 15$ или $E_{phbgo} > 15$}
    \end{itemize}
\end{minipage}
\hfill
\end{figure}
\end{frame}

\begin{frame}\label{r2}
\frametitle{Результаты}
\begin{figure}[h]
  \begin{minipage}[h]{0.49\linewidth}
    \center\textbf{Моделирование \\(840 МэВ, ISR on)}
    \center{\includegraphics[width=0.8\linewidth]{Present/images/xiafterM.png}}
  \end{minipage}
  \hfill
  \begin{minipage}[h]{0.49\linewidth}
    \center\textbf{Эксперимент \\(840 МэВ)}
    \center{\includegraphics[width=0.8\linewidth]{Present/images/xiafter.png}}
  \end{minipage}
  \center{Зависимость коэффициента $\xi_{-}$ от коэффициента $\xi_{+}$}
\end{figure}
\end{frame}

\begin{frame}\label{r2}
\frametitle{Результаты}
\begin{figure}[h]
  \begin{minipage}[h]{0.49\linewidth}
    \center\textbf{Моделирование \\(840 МэВ, ISR on)}
    \center{\includegraphics[width=0.8\linewidth]{Present/images/ks:piSim.png}}
  \end{minipage}
  \hfill
  \begin{minipage}[h]{0.49\linewidth}
    \center\textbf{Эксперимент \\(840 МэВ)}
    \center{\includegraphics[width=0.8\linewidth]{Present/images/ks:piExp.png}}
  \end{minipage}
  \center{Зависимость массы $K_{S}$-мезона от массы $\pi^{0}$-мезона}
\end{figure}
\end{frame}

\begin{frame}
\frametitle{Эффективность регистрации}
\center $\epsilon = {N_{det}\over{N}}$, где $N=10^{5}$.

2011 --- 1.0 Тл, 2012 --- 1.3 Тл
\begin{figure}[h]
    \center{\includegraphics[width=0.6\linewidth]{Present/images/effiency.png}}
\end{figure}
\end{frame}

\begin{frame}
\frametitle{Вычисление сечения процесса}
\begin{figure}[h]
\center{\includegraphics[width=0.5\linewidth]{Present/images/fitksminv.png}}

    Функции подгонки имели следующий виде:
    \begin{itemize}
    \tiny{
    \item $S(x) = N*(\alpha*gausn(x, \bar{x}_{1}, \sigma_{1}) + \beta*gausn(x, \bar{x}_{2}, \sigma_{2}) + (1-\alpha-\beta)*gausn(x, \bar{x}_{3}, \sigma_{3})))$ --- сигнальные события
    \item $\Phi(x) = k*(x-b)$ --- фоновые события
    \item $S(x) + \Phi(x)$ --- итоговая форма. 
    }
    \end{itemize}
\end{figure}
\end{frame}

\begin{frame}
\frametitle{Предварительное сечение}
\center $\sigma = {N\over{\epsilon L}}$, где $N$ --- число отобранных событий, $\epsilon$ --- эффективность регистрации и $L$ --- интегральная светимость.\\
\begin{figure}[h]
    \center\textbf{}
    \center{\includegraphics[width=0.55\linewidth]{Present/images/cross.png}}
\end{figure}
\end{frame}

\begin{frame}
\frametitle{Фоновые процессы}
Основные фоновые процессы
\begin{itemize}
    \item При энергии ниже 1300 МэВ --- $K_{S}K_{L}\gamma$
    \item При энергии выше 1300 МэВ и ниже 1800 МэВ --- $\pi^{+}\pi^{-}\pi^{0}\pi^{0}$
    \item При энергии выше 1800 МэВ --- $K_{S}K_{L}\pi^{0}\pi^{0}$
\end{itemize}
\end{frame}

\begin{frame}
\frametitle{Заключение и планы}
Заключение:
\begin{itemize}
    \item Выработаны оптимальные критерии отбора
    \item Определена эффективность регистрации
    \item Получено предварительное сечение процесса $e^{+}\:e^{-} \to K_{S}\:K_{L}\:\pi^{0}$
%    \item Сделана оценка вклада фоновых процессов (< 10\%)
\end{itemize}
Планы:
\begin{itemize}
    \item Более детальное изучение фоновых процессов
    \item Подготовка публикации
\end{itemize}
\end{frame}

\begin{frame}
\frametitle{Ионизационные потери}
\begin{figure}[h]
\center\textbf{Ионизационные потери положительно-заряженных частицы слева и отрицательных справа}
 \begin{minipage}[h]{0.49\linewidth}
    \center{\includegraphics[width=0.8\linewidth]{Present/images/dedxplus.png}}
  \end{minipage}
  \hfill
  \begin{minipage}[h]{0.49\linewidth}
    \center{\includegraphics[width=0.8\linewidth]{Present/images/dedxminus.png}}
  \end{minipage}
   \small\center{$N_{tr} = 2$, $N_{ph} \geq 2$, $N_{K_{S}} = 1$}
\end{figure}
\end{frame}

\begin{frame}
\frametitle{Энергия и Масса $K_{L}$-мезона}
\begin{figure}[h]
\center\textbf{Распределение энергий слева и распределение масс справа при энергии 1900 МэВ}\\
 \begin{minipage}[h]{0.49\linewidth}
    \center{\includegraphics[width=1\linewidth]{Present/images/klenergyExp.png}}
  \end{minipage}
  \hfill
  \begin{minipage}[h]{0.49\linewidth}
    \center{\includegraphics[width=1\linewidth]{Present/images/klminvExp.png}}
  \end{minipage}
\end{figure}
\end{frame}

\begin{frame}
\frametitle{Результаты\\(840 МэВ)}
\begin{figure}[h]
    \center{\includegraphics[width=0.6\linewidth]{Present/images/rho_mafter.png}}
\end{figure}
\end{frame}
\end{document} 