\chapter*{Литературный обзор} \label{lit_rev}
\addcontentsline{toc}{chapter}{Литературный обзор}	% Добавляем его в оглавление

В этой главе будут разобраны публикации и источники, на которых основывается представленная работа. В абзацах будут даны ссылки на работы, которые дают общее представление о источниках синхротронного излучения и их применение, и далее, более конкретно, ссылки на статьи касательно теории и методов моделирования частично когерентного СИ. \rr{...}

Общее введение в теорию источников синхротронного излечения и их применение может быть найдено в ряде книг:~\cite{willmott_introduction_2019},~\cite{alsnielsen_elements_2011},~\cite{onuki_undulators_2003}. В ~\cite{willmott_introduction_2019} и~\cite{alsnielsen_elements_2011} даются общие представления о источниках синхротронного излучения, и основных компонентах на рабочих станциях, разбирается теоретическая основа и практическое применение основных методик реализуемых с помощью синхротронного излучения (в частности, в рентгеновском диапазоне длин волн). В книге~\cite{onuki_undulators_2003} даётся введение в динамику электронного пучка в накопительных кольцах, устройство вставных устройств: поворотных магнитов, вигглеров и ондуляторов. Более глубокие 
разъяснения касательно ускорительной техники могут быть найдены в~\cite{wiedemann_particle_2015}.

Общих подход при моделировании распространения волнового фронта через оптическую линию рабочей станции основывается на подходах Фурье-оптики~\cite{goodman_introduction_2005}

\rr{Надо написать кем и когда было предсказано синхротронное излучение, когда было впервые наблюдено. Указать по датам развитие синхротронных источников излучения от 1 до 3 поколения. Показать новый milestone -- источники 4 поколения, появление дифракционно ограниченных источников. Показать развитие кодов для моделирования синхротронного излучения: от программ для ray-tracing до wavefront propagation от Чубаря. Указать, как происходит моделирование. Показать основыне методы моделирования частично когерентного излучения. Разложение по Гаусс-Шелл модам...написать, что не так с этим подходом (ссылка на статью Джанлуки). Рассказать, что излучение следует гауссовой статистике. }
%============================================================================================================================
