
\chapter{Методы моделирования ондуляторного излучения от пучка с конечным эмиттансом} \label{chapt2}
При обсуждении методов моделирования будет рассматриваться только ондуляторное излучение. Отчасти это мотивировано относительной простотой рассмотрения ондуляторного излучения, по сравнению, например, с вигглерным \cite{geloni_brightness_2014}. С другой стороны, при рассмотрении излучения источников четвёртного поколения, излучение из ондуляторов обладает высокой степенью поперечной когерентности, что и представляет интерес в рамках представленной работы. Тем не менее, формула~\ref{eq:E_bunch} применима и для вигглерного излучения и для излучения поворотного магнита. Для расчёта конечного поля по формуле~\ref{eq:E_bunch}, необходимо сгенерировать $N_e$ полей от каждого электрона и сложить их. Таким образом получится одна реализация поля, далее необходимо повторить операцию генерации поля для $N_b$ реализаций и усреднить модули квадратов полей по получившемуся статистическому ансамблю. Метод является крайне медленным, поэтому в случае продольно некогерентного источника подойдёт метод реализованный в коде SRW \cite{chubar_accurate_1998}, \cite{chubar_simulation_2006}. Он основан на сложение интенсивностей полей от каждого электрона, где отсутствует необходимость усреднять по статистическому ансамблю, что ускоряет процесс вычисления. Оба метода будут обсуждаться в настоящей Главе. 

Альтернативный подход в моделировании частично когерентного излучения основывается на декомпозиции функции взаимной когерентности синхротронного излучения на Гаусс-Шелл моды (разложение по полиномам Эрмита) описываемый в работах \cite{singer_modelling_2011}, \cite{hua_application_2012}, \cite{khubbutdinov_coherence_2019}, \cite{noauthor_iucr_nodate}. Однако, как отмечают сами авторы в \cite{khubbutdinov_coherence_2019}, \cite{noauthor_iucr_nodate} и аналитически описывается в \cite{geloni_transverse_2008}, разложение по полиномам Эрмита не применимо в случае, когда источник имеет высокую степень когерентности. Так как область когерентности на источнике весьма высока, то функции, которые описывают поведение ондуляторного излучения и в дальней зоне, и на источнике, имеют не гауссову природу.

В текущей главе описывается новый алгоритм моделирования синхротронного излучения, для краткости называемый СЕРВАЛ. Алгоритм основывается на прямом моделировании стохастических процессов при генерации синхротронного излучения, вызванных дробовым шумом в электронном пучке, с последующим ограничением пространственных гармоник шума огибающими излучения. По своей природе алгоритм имеет оценочный характер, именно поэтому в главе приведён сравнительный анализ результатов СЕРВАЛа с методом сложения амплитуд, на примере некоторых оптических систем. СЕРВАЛ показал себя как мощный инструмент для оценки когерентных свойств синхротронного излучения, точность которого мало уступает методу сложения амплитуд, но имеющий преимущество в быстродействии. 
\section{Расчёт ондуляторного излучения прямым моделированием методом Монте-Карло}
Формула~\ref{eq:E_bunch} используется напрямую при моделирования монохроматизированного ондуляторного излучения, как продольно когерентного, так и некогерентного. Общий вид поля ондуляторного излучения от одного электрона с некоторыми углом $\vec{\eta}_k$ и координатой $\vec{\l}_k$ может быть записан как \cite{geloni_fourier_2007}: 
\begin{align}
	\bar{E}_{\bot}(z_0, \omega, \vec{\eta}_k, \vec{l}_k, \vec{\theta} \;) =
&	-\frac{\omega e A_{JJ} L_w}{2c^2z_0}\frac{K}{\gamma}\exp{\bigg[i \frac{\omega z_0}{2c} \bigg|\vec{\theta} - \vec{\l}/z_0\bigg|^2 \bigg]} \cr & \times \sinc \bigg[\frac{\omega L_s |\vec{\theta} - (\vec{\l}/z_0) - \vec{\eta}|^2}{4c}\bigg],
	\label{eq:single_electron_far_field}
\end{align}
где $\vec{\theta} = \vec{r}/z_0$, $e$ -- заряд электрона, $\gamma$ -- Лоренц-фактор, $K$ -- параметр ондуляторности, $k_w = 2 \pi / \lambda_w$, где $\lambda_w$ -- период ондулятора, $\Delta \omega$ -- отстройка от резонансной частоты. $A_{JJ} = J_0(\zeta) - J_1(\zeta)$, $\zeta = K^2/(4 + 2K^2)$ и $J_n$ -- функция Бесселя первого рода $n$ порядка. Формула \ref{eq:single_electron_far_field} даёт распределение амплитуды поля в дальней зоне. Определение дальней зоны для ондуляторного излучения обсуждается в~\cite{geloni_fourier_2007}. Чтобы получить точное выражение, это поле должно быть распространено назад в центр ондулятора с помощью пропагатора\footnote{В дальнейшем изложении вместо слова «распространение» будет использоваться слово «пропогация»} свободного пространства~\cite{voelz_computational_2011},~\cite{schmidt_numerical_2010}. Распределение поля в мнимом источнике излучения:
\begin{align}
	\bar{E}_{\bot}(0, \omega, \vec{\eta}, \vec{\l}, \vec{r}_{\bot}) =
	i \frac{e A_{JJ} \omega}{2 c^2}\frac{K}{\gamma} &\exp{\big[i \frac{\omega}{c} (\vec{r}_{\bot} - \vec{l})\big]}\cr & \times \bigg [\pi - 2\text{Si} \bigg( \cfrac{i \omega |\vec{r}_{\bot} - \vec{l}|^2}{L_w c}\bigg)\bigg], 
	\label{eq:single_electron_near_field_z=0}
\end{align}
после этого поле можно распространять на любую дистанцию вдоль оптической оси $z_0$, что показано в~\cite{geloni_fourier_2007}, снова применяя пропагатор свободного пространства:
\begin{align}
	\bar{E}_{\bot}(&z_0, \omega, \vec{\eta}_k, \vec{\l}_k, \vec{r}) =
		\frac{e A_{JJ} \omega}{2 c^2}\frac{K}{\gamma} \exp{\bigg[i \frac{\omega}{2 z_0 c} (|\vec{r}_{\bot} - \vec{l}|^2 - |\vec{r}_{\bot} - \vec{l} - z_0 \vec{\eta}|^2) \bigg]} \cr & \times	\bigg \{ \text{Ei} \bigg[ \cfrac{i \omega (\vec{r}_{\bot} - \vec{l} - z_0 \vec{\eta})^2}{2z_0c - L_w c}\bigg] - \text{Ei} \bigg[ \cfrac{i \omega (\vec{r}_{\bot} - \vec{l} - z_0 \vec{\eta})^2}{2z_0c + L_w c} \bigg] \bigg\}.
	\label{eq:single_electron_near_field}
\end{align}
Рассчитанное таким образом поле может быть использовано для любого значения $z_0$, кроме точки $z_0 = L_w/2$ и $r_{\bot} = 0$, см.~\cite{geloni_fourier_2007}. Обе формулы \ref{eq:single_electron_far_field} и \ref{eq:single_electron_near_field} имеют практическую ценность при моделировании. Стоит заметить, что при использовании выражения \ref{eq:single_electron_near_field} время моделирования значительно увеличивается, так как необходимо дважды численно взять интеграл $\textup{Ei}(\cdot)$.
 
\subsection{Метод сложения амплитуд}
После расчёта суммарного поля c $N_e$ электронами по формуле~\ref{eq:E_bunch}, получившиеся монохроматическое поле по своей сути есть одна статистическая реализация. Физически это значит следующее: если экспериментатор измерит распределение интенсивности поля на детекторе от пролёта одного электронного пучка, используя монохроматор с разрешением, которое позволит разрешить одну продольную моду излучения, то на детекторе будет распределение, эквивалентное по своим свойствам представленному на Рис.~\ref{fig:spikes2}. 
\begin{figure}[H]
	\centering 	\includegraphics[width=0.7\linewidth]{spikes2.png}
	\caption{Спайковая структура излучения синхротронного излучения. Красной линией обозначена разрешающая способность монохроматора}
	\label{fig:spikes2}
\end{figure}
После усреднения по $N_b$ реализациям (с идеальным монохроматором\footnote{другими словами, монохроматором разрешается одна поперечная мода}), наблюдаемая интенсивность даётся выражением: 
\begin{align}
	I_{\omega} = \bigg \langle \bigg|\sum\limits_{k=1}^{N_e} \bar{E}(\vec{\eta}_k, \vec{\l}_k, z, \vec{r}, \omega) \exp{(i \omega t_k)}\bigg|^2 \bigg \rangle,
	\label{eq:I_MC} 
\end{align}
\noindent результирующая интенсивность будет сходиться к некоторой огибающей. В грубом приближении огибающая является свёрткой распределения расходимости излучения с распределением расходимости электронного пучка. Общая схема метода сложения амплитуд изображена на Рис.~\ref{fig:MC_scheme}. 
\begin{figure}[H] 
	\centering 	\includegraphics[width=0.99\linewidth]{MC_scheme.png}
	\caption{Схема работы метода сложения амплитуд}
	\label{fig:MC_scheme}
\end{figure}
Данный подход является наиболее прямым подходом в задаче моделирования частично когерентного излучения, однако время расчёта в таком случае может быть оценено как время затрачиваемое на расчёт одной одного поля $N_e$ раз по формуле~\ref{eq:single_electron_far_field} или~\ref{eq:single_electron_near_field}, в последней, как уже упоминалось, необходимо дважды численно взять интеграл $\textup{Ei}(\cdot)$ и потом усреднить по $N_b$ реализациям поля $\bar{E}_{b}$. Итого, если за $\tau_{calc}$ взять время расчёта одного поля формуле~\ref{eq:single_electron_far_field} или~\ref{eq:single_electron_near_field}, то расчёт одного результирующего поля $\bar{E}_{b}$ в сумме займёт $T_{calc} = \tau_{calc} \cdot N_e \cdot N_b$.

\subsection{Метод сложения интенсивностей}
В случае полностью некогерентного излучения время расчёта можно сократить за счёт фазового фактора $\exp{(i \omega t_k)}$, который эффективно приводит к тому, что излучение от отдельно электрона в электронном пучке коррелирует только с самим собой \cite{geloni_transverse_2008}. Если расписать выражение~\ref{eq:I_MC}. получим: 
\begin{align}
	I_{\omega} = \sum\limits_{k=1}^{N_e} \bar{E}(\vec{\eta}_k, \vec{\l}_k, z, \vec{r}, \omega)\bar{E}^{*}(\vec{\eta}_k, \vec{\l}_k, z, \vec{r}, \omega) + \cr
	\bigg \langle \sum\limits_{k=1}^{N_e} \sum\limits_{n=1}^{N_e} \bar{E}(\vec{\eta}_k, \vec{\l}_k, z, \vec{r}, \omega)\bar{E}^{*}(\vec{\eta}_n, \vec{\l}_n, z, \vec{r}, \omega) & \exp{\big[i \omega (t_k - t_n)\big]}\bigg \rangle,
	\label{eq:I_SRW_with_explicit_sum} 
\end{align}
в котором после усреднения второе слагаемое будет равно нулю из-за упомянутого фазового фактора. Таким образом формула~\ref{eq:I_MC} упрощается до 
\begin{align}
 	I_{\omega} = \sum\limits_{k=1}^{N_e} \bigg|\bar{E}(\vec{\eta}_k, \vec{\l}_k, z, \vec{r}, \omega)\bigg|^2,
 	\label{eq:I_SRW} 
\end{align}
а время расчёта уменьшается до $T_{calc} = \tau_{calc} \cdot N_e$. Этот метод, для условности, будет носить название метод сложения интенсивностей. Общая схема метода представлена на Рис.~\ref{fig:SRW_scheme}.
\begin{figure}[H] 
	\centering 	\includegraphics[width=0.99\linewidth]{SRW_scheme.png}
	\caption{Схема метода сложения интенсивностей}
	\label{fig:SRW_scheme}
\end{figure}
\noindent  Недостатком такого подхода можно считать потерю фазовой информации об излучении и, следовательно, невозможности расчёта функции взаимной когерентности первого порядка. Тем не менее, подход основанный на формуле~\ref{eq:I_SRW} даёт мощный метод расчёта наблюдаемых интенсивностей для частично когерентного излучения. Именно этот подход реализован в широко распространённом коде SRW.

В заключении к двум предыдущим разделам отдельно необходимо отметить: $N_b$ -- это физическая величина. Если разрешить монохроматором ровно одну продольную моду и набрать статистику, например, из $10$ электронных пучков, полученная интенсивность на детекторе будет, очевидно, соответствовать 10 реализациям поля. Иначе дело обстоит с числом макроэлектронов -- $N_e$. Для достоверного моделирования поперечной спайковой структуры синхротронного излучения необходимо взять число $N_e$ как минимум больше, чем характерное количество мод. Иначе, мелкие детали спайковой структуры просто на разрешаться.
  
\section{Учёт влияния размера электронного пучка на расходимость излучения при помощи прямых методов Монте-Карло}
При помощи метода сложения амплитуд, можно получить поля, в которых видно влияние продольной когерентности\footnote{в смысле установленном в Главе~\ref{chapt1}} и размера электронного пучка на расходимость излучения.
\subsection{Влияние размера электронного пучка на расходимость излучения}
Первый эффект -- влияние размера электронного пучка на расходимость излучения и, следовательно на поперечный размер излучения в дальней зоне. Этот эффект обсуждается в работе~\cite{chubar_simulation_2006} разработчиком кода SRW применительно к когерентному синхротронному излучению (англ. coherent synchrotron radiation (CSR)). Под CSR подразумевает продольно когерентное излучение, реализуемое, когда электронный пучок много меньше излучаемой длины волны. На примере CSR можно наблюдать следующий эффект: если электронный пучок меньше или сравним с размером перетяжки излучения на источнике, наблюдается обычная расходимость излучения, определяемая свёрткой натуральной расходимости синхротронного излучения с расходимостью электронного пучка.
\begin{figure}[h!]
	\centering 	\includegraphics[width=0.99\linewidth]{diff_divergence_coh_scheme.png}
	\caption{Схема эффекта зависимости расходимости излучения от размера электронного пучка. Рисунок разделён на две части: ближняя зона, источник излучения -- слева и дальняя зона -- справа. Жёлтой линией схематично изображена характерный поперечный размер излучения, синей линей характерный поперечный размер электронного пучка, жёлтая волнистая линия символизирует длину волны излучения в сравнение с продольным размером электронного пучка. Важно отметить, схематичный масштаб для обоих рисунков (верхнего и нижнего), -- тот же. Расходимости электронных пучков (сверху и снизу) одинаковы.}
	\label{fig:diff_divergence_coh_scheme}
\end{figure}
Однако, при увеличении размера электронного пучка, при той же расходимости, наблюдается эффект уменьшения расходимости излучения. Как отмечает автор в~\cite{chubar_simulation_2006}, этот эффект объясняется с точки зрения гауссовой оптики: при увеличении размера источника, угловой размер должен уменьшаться, формула~\ref{eq:photons_emittance}. На Рис~\ref{fig:diff_divergence_coh_scheme} изображена схема описывающая этот эффект.

При расчёте ондуляторного излучения методом сложения амплитуд этот эффект выглядит следующим образом, Рис.~\ref{fig:diff_coh_incoh_rad}.
\begin{figure}[H] 
	\centering 	\includegraphics[width=0.5\linewidth]{diff_divergence_coh.png}
	\caption{Расходимость излучения от электронного пучка с размерами, указанными в легенде. Расходимость электронного пучка много меньше натуральной расходимости синхротронного излучения}
	\label{fig:diff_coh_incoh_rad}
\end{figure}
Расчёт проводился для модельных параметрах электронного пучка: расходимость была взята много меньшей чем натуральная расходимость ондуляторного излучения, размеры электронного пучка указаны в легенде к Рис.~\ref{fig:diff_coh_incoh_rad}, резонансная энергия на $12,4$ эВ, ондулятор с $200$ периодами, длина периода $18$ мм. Синяя линия на Рис.~\ref{fig:diff_coh_incoh_rad} отвечает случаю электронного пучка с бесконечно малым эмиттансом.

\subsection{Различие расходимости излучения для случая продольно когерентного и некогерентного излучения}
В зависимости от длительности электронного пучка результирующее поле $\bar{E}_{b}$ будет вести себя по-разному. В случае короткого электронного пучка: $\omega \sigma_T \ll 1$, где $\sigma_T$ -- длительность электронного сгустка, излучение будет продольно когерентным, а в случай длинного электронного пучка, а именно  $\omega \sigma_T \gg 1$, соответствует продольно некогерентному излучению. 

\begin{figure}[H] 
	\centering 	\includegraphics[width=0.99\linewidth]{coh_incoh_divergence.png}
	\caption{Схема эффекта зависимости расходимости излучения от продольной когерентности излучения. Жёлтой линией обозначен характерный поперечный размер излучения, жёлтой волнистой длина волны излучения с сравнение в продольным размером электронного пучка. На рисунке обозначено, что в случае продольно некогерентного излучения расходимость в $\sqrt{2}$ раз больше, чем в случае продольно когерентного излучения}
	\label{fig:coh_incoh_divergence}
\end{figure}
Расчёт проводился дли гипотетического случая электронного пучка с размерами пучка много меньшими натуральных размеров излучения в перетяжке\footnote{Чтобы для когерентного случая избежать эффекта, описанного в предыдущем разделе}, $\sigma_x' = 20$ мкрад, $\sigma_y' = 20$ мкрад на резонансной энергии $300$ эВ. Ондулятор с $200$ периодами, длина периода $18$ мм. 

\begin{figure}[H] 
	\centering 	\includegraphics[width=0.5\linewidth]{diff_coh_incoh_rad.png}
	\caption{Расходимость излучения в случае продольно когерентного излучения (зелёная линия), и продольно некогерентного излучения (синяя линия).}
	\label{fig:diff_coh_incoh_rad}
\end{figure}
\noindent Этот эффект, по всей видимости, не обсуждался в литературе, однако заслуживает дальнейшего исследования.
\section{Метод ограничения пространственных гармоник огибающими: СЕРВАЛ}
Предлагаемый алгоритм основывается на моделировании стохастического характера ондуляторного синхротронного излучения комплексным гауссовым шумом с последующим его ограничением огибающими поля. Алгоритм описывает продольное некогерентное ондуляторное излучение. Для начала алгоритм будет представлен в общем виде, без уточнения чем определяются распределения огибающих, задающих размер и расходимость излучения. 
\subsection{Алгоритм создания поля}
Алгоритм выполняется в три этапа: создание комплексного гауссового шума; его ограничение размерами излучения в перетяжке в $r$-пространстве; его ограничение расходимости излучения в $k$-пространстве\footnote{Излучение от всего электронного пучка.}. Полное описание алгоритма приведено ниже: 
\begin{enumerate}
\item \label{noise} Создание комлексного гауссового шума $Z = X + iY$ в $r\omega$-пространстве, где величины $X$ и $Y$ подчиняются нормальному распределению.
\begin{figure}[H] 
	\centering 	\includegraphics[width=0.45\linewidth]{1-X_noise.png}
	\caption{Интенсивность комплексного гауссового шума}
	\label{fig:1-noise}
\end{figure}
\item \label{beam_s} Ограничение шума эффективным размером электромагнитного излучения в источнике излучения в \textit{$r$}-пространстве.
\begin{figure}[H]
	\centering
	\begin{minipage}{0.45\textwidth}
		\centering
		\includegraphics[width=1\linewidth]{2-X_e-beam-size.png}
		\caption{Излучение с наложенным шумом}
		\label{fig:2-beam_size_k}
	\end{minipage}
	\begin{minipage}{0.45\textwidth}
		\centering
		\includegraphics[width=1\linewidth]{2-X_e-beam-divergence.png}
		\caption{Моды излучения в $k$-пространстве на источнике}
		\label{fig:2-beam_size_s}
	\end{minipage}\hfill
\end{figure}
\item \label{beam_k} Ограничение пространственных мод эффективной расходимостью излучения в \textit{k}-пространстве
\begin{figure}[H]
	\centering
	\begin{minipage}{0.45\textwidth}
		\centering
		\includegraphics[width=1\linewidth]{3-X_radaition_divergence.png}
		\caption{Расходимость излучения в источнике}
		\label{fig:3-beam_s}
	\end{minipage}
	\begin{minipage}{0.45\textwidth}
		\centering
		\includegraphics[width=1\linewidth]{3-X_radaition_size.png}
		\caption{Размер излучения в источнике}
		\label{fig:3-beam_k}
	\end{minipage}
\end{figure}
\end{enumerate}
\noindent Получившиеся распространение поля есть распределение поля в источнике излучения (центр ондулятора). Излучение монохроматично, т.е. продольно разрешён один спайк.

Быстродействие алгоритма можно оценить следующим образом: алгоритм генерирует $N_x \cdot N_y \cdot N_b$ случайных величин, подчиняющихся распределению $Z$, где $N_b$ -- количество реализаций поля, совершает одного преобразования Фурье поля (преобразование поля на Рис.~\ref{fig:2-beam_size_k} в поле на Рис.~\ref{fig:2-beam_size_s}) и два раза умножает на огибающие поля. Получившиеся поле, представленное на Рис.~\ref{fig:3-beam_k}, уже готово к распространению, так как пропагатор через свободное пространство работает именно в $kf$-пространстве. 

Показательно сравнить быстродействие разобранных алгоритмов (Таблица~\ref{tab:speed}). При сравнение использовалась поперечная сетка $N_x \times N_y = 501 \times 501$, для расчёта поля в дальней зоне использовалась формула~\ref{eq:single_electron_far_field}. Сравнение производилось для разного количества макроэлектронов в методе сложения амплитуд.
\begin{table}[H]
	\caption{Быстродействие метода сложения амплитуд (МСА) и СЕРВАЛа}
	\label{tab:speed}	
	\begin{tabular}{l|c|c|c}
	$N_e$ & МСА, сек/реализация &СЕРВАЛ, сек/реализация \\ 
	\hline
	100   & 2.8 & 0.020\\
	\hline 
	200   & 5.5 & 0.020 \\
	\hline 
	400   & 11  & 0.020 \\
	\end{tabular}
\end{table} 
Быстродействие метода сложения амплитуд и СЕРВАЛа сравнивается напрямую, так как в обоих методах есть необходимость усреднять по реализациям. Однако, для метода сложения интенсивностей нет понятия реализации поля, если только количество макроэлектронов. Для прямого сравнения СЕРВАЛА и метода сложения интенсивностей необходимо сравнивать быстроту генерации одного поля. Для той же плотности точек поперечной сетки, при равном числе реализаций и количестве макроэлектронов -- 400, расчёт поля СЕРВАЛом занял \textbf{8 сек}, методом сложения интенсивностей\footnote{Здесь, снова, стоит отметить, при использовании метода сложения интенсивностей теряется вся фазовая информация о поле, метод позволяет моделировать только наблюдаемые интенсивности полей.} \textbf{15 сек}. 
\subsection{Выбор подходящих огибающих}
До этого момента в работе не обсуждался выбор подходящих огибающих для поля, генерируемого СЕРВАЛом. Вопрос выбора таких огибающих сводится к нахождению распределения поля в центре ондулятора. Поле в центре ондулятора может быть получено обратным распространением излучения из дальней зоны обратно в центр ондулятора при помощи пропагатора излучения в свободном пространстве. Однако, нахождение аналитического решения уравнения Максвелла в дальней зоне от целого электронного пучка -- не тривиальная задача.  Для оценки можно предположить, что распределение поля ондуляторного излучения от электронного пучка с конечным эмиттансам, в целом, может быть представлено как свёртка распределения поля ондуляторного излучения от одного электрона с распределением фазового пространства электронного пучка~\cite{geloni_transverse_2008},~\cite{chubar_simulation_2006}.

Для СЕРВАЛа можно предложить, как минимум, три вида огибающих для пространственного распределения источника в $r$-пространстве:
\begin{enumerate}[label=\Roman*.]
	\item \label{amplitude} ${A}_{b} (\vec{r}_{\bot}) = \big(\bar{E}_{\bot}(0, \vec{l}, \vec{\eta}, \vec{r}_{\bot}) \ast f_l(\vec{l})\big)(\vec{l})$ \\

	\item \label{intensity} ${A}_{b} (\vec{r}_{\bot}) = \sqrt{\big(\bar{E}^2_{\bot}(0,  \vec{l}, \vec{\eta}, \vec{r}_{\bot}) \ast f_l^2(\vec{l})\big)(\vec{l})}$ \\

	\item \label{e-beam} ${A}_{b} (\vec{r}_{\bot}) = f_l(\vec{l})$,
\end{enumerate}
и три вида огибающих для распределения расходимости источника -- $k$-пространство:
\begin{enumerate}[label=\Roman*.]
	\item \label{amplitude} $\hat{{A}}_{b} (\vec{\theta}_{\bot}) = \big(\hat{\bar{E}}_{\bot}(0,  \vec{l}, \vec{\eta}, \vec{\theta}_{\bot}) \ast \hat{f}_{\eta}(\vec{\eta})\big)(\vec{\eta})$\\
	
	\item \label{intensity} $\hat{{A}}_{b} (\vec{\theta}_{\bot}) = \sqrt{\big(\hat{\bar{E}}^2_{\bot}(0,  \vec{l}, \vec{\eta}, \vec{\theta}_{\bot}) \ast \hat{f_{\eta}^2}(\vec{\eta})\big)(\vec{\eta})}$\\
	
	\item \label{e-beam} $\hat{{A}}_{b} (\vec{\theta}_{\bot}) = \hat{f}_{\eta}(\vec{\eta})$,
\end{enumerate}
где ${A}_{b} (\vec{r})$ и $\hat{{A}}_{b} (\vec{\theta})$ огибающие в $r$- и $k$-пространствах, соответствующие шагам~\ref{beam_s} и~\ref{beam_k} в алгоритме,  $f(\vec{l}, \vec{\eta}) = f_l(\vec{l}) f_{\eta}(\vec{\eta})$ распределение фазового пространства электронного пучка, и поле $\bar{E}_{\bot}(z=0, \vec{\l}, \vec{\eta}, \vec{r}_{\bot})$, $\hat{\bar{E}}_{\bot}(z=0 , \vec{\l}, \vec{\eta}, \vec{\theta}_{\bot})$ -- распределение поля, взятого в центре ондулятора по формулам \ref{eq:single_electron_far_field}(или более точно \ref{eq:single_electron_near_field}) и \ref{eq:single_electron_near_field_z=0}.

При выборе подходящих амплитуд было проведено моделирование с различными огибающими в сравнении с эталонным в этой работе методом сложения амплитуд. Для начала необходимо проверить распределение интенсивности поля на источнике. Для метода сложения амплитуд поле было рассчитано в дальней зоне и распространено назад в центр ондулятора. Результаты сравнения приведены на Рис.~\ref{fig:SERVAL_envelopes_comparison_far_zone} и представлены на Рис.~\ref{fig:SERVAL_envelopes_comparison_source}, Рис.~\ref{fig:SERVAL_corr_comparison}, Рис.~\ref{fig:SERVAL_envelopes_comparison_far_zone}. В работе использовались следующие параметры: для ондулятора Таблица.~\ref{tab:undulator_parameters}
\begin{table}[H]
	\caption{Параметры ондулятора}
	\label{tab:undulator_parameters}	
	\begin{tabular}{l|c|r}	
		$E_{ph},  [\textup{эВ}]/\lambda, [\textup{\AA}$]& $\lambda_w, [\textup{мм}]$ & периодов\\ 
		\hline	%0.65-1.35
		2167/5.72    &  18      & 200   
	\end{tabular}
\end{table}
\noindent Расчёты были проведены с использование параметров электронного пучка ЦКП «СКИФ» для одного из прямых промежутков (Таблице~\ref{tab:SKIF parameters}).
\begin{table}[H]
	\caption{Параметры электронного пучка}
	\label{tab:SKIF parameters}	
	\begin{tabular}{l|c|c|c|r}
		$E, \textup{[GeV]}$ & $\sigma_x, \textup{[мкм]}$ & $\sigma_y, \textup{[мкм]}$ & $\sigma_{x'}, \textup{[мкрад]}$ & $\sigma_{y'}, \textup{[мкрад]}$ \\ 
		\hline
		3          &38                          & 4.7                        & 25                          & 20 
	\end{tabular}
\end{table} 
\begin{figure}[H] 
	\centering 	\includegraphics[width=0.99\linewidth]{SERVAL_envelopes_comparison_source.png}
	\caption{Распределение поля на источнике излучения для разных огибающих в сравнении с распределением, даваемым методом сложения амплитуд}
	\label{fig:SERVAL_envelopes_comparison_source}
\end{figure}
\noindent Видно, что оптимальные результаты достигаются при использовании свёртки~\ref{intensity} Однако, если размер электронного пучка много больше или даже сравним с натуральным размером излучения в перетяжке, то можно использовать любые из представленных огибающих для $r$-пространства. Необходимо так же сравнить корреляционные функции получившихся полей~\ref{fig:diff_coh_incoh_rad} в дальней зоне на 25 метрах от ондулятора, используя формулу~\ref{eq:g1}.
\begin{figure}[H] 
	\centering 	\includegraphics[width=0.99\linewidth]{SERVAL_corr_comparison.png}
	\caption{Функция взаимной когерентности на расстоянии $25$ м от источника}
	\label{fig:SERVAL_corr_comparison}
\end{figure}
\noindent Для распределения расходимости следует так же использовать свёртку интенсивностей.
\begin{figure}[H] 
	\centering 	\includegraphics[width=0.99\linewidth]{SERVAL_envelopes_comparison_far_zone.png}
	\caption{Размеры излучения в дальней зоне (расходимость) на расстоянии $25$ м}
	\label{fig:SERVAL_envelopes_comparison_far_zone}
\end{figure}
В большинстве случаев можно выбирать свёртку интенсивностей по~\ref{intensity} Однако, стоит отметить, что СЕРВАЛ -- это оценочный метод и, в случае дифракционного ограниченного источника, необходимо перед проведением расчётов сделать подобный анализ подходящих огибающих. Для примера, в Приложении~\ref{AppendixA} дан подобный анализ для электронных пучков в различных приближениях.

\chapter{Применение СЕРВАЛа} \label{chapt3}
СЕРВАЛ является эффективным алгоритмом для моделирования частично когерентного синхротронного излучения, в случаях когда есть заметная степень когерентности источника излучения. Уже было показано совпадение распределений интенсивности в дальней зоне и на источнике излучения, а так же совпадение корреляционных функций с методом сложения амплитуд, который может считаться методом, дающим результат «из первых принципов» во всех ситуациях\footnote{необходимо помнить, что число $N_e$ должно быть достаточно велико для получения достоверного результата}. Этот сравнительный анализ свойств источника излучения показывает, что весьма ресурсозатратный по времени метод сложения амплитуд может быть заменён СЕРВАЛом без потери точности и физичности результатов. В этой главе мы приведём ещё один сравнительный анализ СЕРВАЛа и метода сложения амплитуд на примере фокусирующей системы с конечной апертурой, а так же два практических применения СЕРВАЛа на примере эксперимента Юнга и нетривиальной задачи отражения частично когерентного излучения от рентгеновского зеркала с шероховатостями. 
Отдельно необходимо отметить, что в случае, когда источник дифракционно ограничен, целесообразно применять метод сложения амплитуд или метод сложения интенсивностей, которые очень быстро дадут сходимость. В этом случае для СЕРВАЛа потребуется тщательный анализ подходящих огибающих и, строго говоря, метод \textit{не моделирует} фундаментальную моду ондуляторного излучения -- случай излучения электронного пучка с бесконечно малым эмиттансом. Для источников с низкой степенью когерентности имеет смысл рассмотреть метод трассировки лучей, потому что все три волновых метода будут иметь весьма низкую сходимость и придётся моделировать большое число статистических реализаций для получения сходимости. В любом случае, прежде чем проводить оптический расчёт, необходимо изучить свойства источника излучения, например, при помощи программы SPECTRA~\cite{tanaka_spectra_2001}, где можно оценить ожидаемую степень когерентности. Только исходя из свойств источника можно применять один из описанных методов моделирования. Именно такой подход даст оптимальный результат в смысле затраченного времени и достоверности полученных результатов. 

\section{Фокусирующая система с конечной апертурой}\label{section:focusing_system_with_aperture}
В представленном разделе будет рассматриваться оптическая схема, состоящая из источника излучения -- ондулятора, апертуры и фокусирующего элемента. Параметры ондулятора и электронного пучка те же, что в Таблицах~\ref{tab:undulator_parameters} и~\ref{tab:SKIF parameters}. Размер апертуры $1 \times 1$ мм$^2$. Для SERVAL были выбраны огибающие~\ref{intensity} Этот расчёт будет сопровождаться сравнением результатов метода СЕРВАЛ с результатами метода сложения амплитуд (МСА).

\begin{figure}[H] 
	\centering 	\includegraphics[width=0.99\linewidth]{beamline.png}
	\caption{Оптическая схема. Ондулятор в начале координат, апертура перед линзой на расстоянии 25 м от ондулятора, линза с фокусным расстоянием 12.5 м фокусирует излучение на образец, расположенный на 25 м от линзы. Распределения интенсивности смотрятся: на источнике, в дальней зоне перед апертурой, на половине пути к образцу и на самом образце.}
	\label{fig:beamline}
\end{figure}
Мнимое распределение интенсивности излучения на источнике представлено на~Рис.~\ref{fig:focusing_system_source}
\begin{figure}[H] 
	\centering 	\includegraphics[width=0.99\linewidth]{0-source3.80E-05_um_4.68E-06_um_2.50E-05_urad_2.00E-05_urad_example_beamline.png}
	\caption{Распределение интенсивности на источнике излучения с величиной среднеквадратичного отклонения $40.8 \times 10.8$ мкм$^2$}
	\label{fig:focusing_system_source}
\end{figure}
\noindent Распределение поля в дальней зоне на 25 м от ондулятора представлено на Рис.~\ref{fig:focusing_system_far_zone}.
\begin{figure}[H] 
	\centering 	\includegraphics[width=0.99\linewidth]{1-far_zone_25_m3.80E-05_um_4.68E-06_um_2.50E-05_urad_2.00E-05_urad_example_beamline.png}
	\caption{Распределение интенсивности излучения в дальней зоне с величиной среднеквадратичного отклонения $750 \times 635$ мкм$^2$.}
	\label{fig:focusing_system_far_zone}
\end{figure}
\noindent Для усреднения было выбрано 300 реализаций, что даёт достаточную сходимость. Однако, в структуре излучения всё ещё видна характерная спайковая структура. Видно, что количество мод в вертикальном направлении меньше, чем в горизонтальном. Их типичный размер говорит о длине поперечной когерентности в соответствующих направлениях. Размер пятна когерентности представлен на Рис.~\ref{fig:focusing_system_corr}.
\begin{figure}[H] 
	\centering 	\includegraphics[width=0.99\linewidth]{corr3.80E-05_um_4.68E-06_um_2.50E-05_urad_2.00E-05_urad_example_beamline.png}
	\caption{Распределение функции взаимной когерентности, построенное по формуле~\ref{eq:g1} с величиной среднеквадратичного отклонения $40 \times 150 $ мкм$^2$.}
	\label{fig:focusing_system_corr}
\end{figure}
\noindent Распределение интенсивности поля после дифракции на апертуре и $12.5$ метрах распространения поля через пустое пространство приведено на Рис.~\ref{fig:focusing_system_after_aperture}
\begin{figure}[H] 
	\centering 	\includegraphics[width=0.99\linewidth]{2-far_zone_12_5_m_after_aperture3.80E-05_um_4.68E-06_um_2.50E-05_urad_2.00E-05_urad_example_beamline.png}
	\caption{Результат дифракции на апертуре и 12.5 м распространения через пустое пространство, величина среднеквадратичного отклонения представленных распределений $225 \times 222$ мкм$^2$}
	\label{fig:focusing_system_after_aperture}
\end{figure}
\noindent Дифракционные картины отличаются для каждого из направлений: для вертикального дифракционные пики более выраженные ввиду большей длины когерентности, для горизонтального направления заметен только первый дифракционный максимум, что говорит о заметно меньшей степени когерентности.

Распределение поля на образце приведено на Рис.~\ref{fig:focusing_system_in_focus}.
\begin{figure}[H] 
	\centering 	\includegraphics[width=0.99\linewidth]{3-70_m_focal_plane3.80E-05_um_4.68E-06_um_2.50E-05_urad_2.00E-05_urad_example_beamline.png}
	\caption{Распределение излучения в фокусе, величина среднеквадратичного отклонения представленных распределений $42 \times 13.5$ мкм$^2$}
	\label{fig:focusing_system_in_focus}
\end{figure}
\noindent Для моделирования было выбрано соотношение плеч фокусирующей системы -- $1:1$. По критерию Рэлея, при размере апертуры $1 \times 1$ мм$^2$, угловая разрешающая способность такой системы $\theta_{diff} = 1.22 \lambda / D  = 0.7$ мкрад, где $D$ размер апертуры. Угловой размер источника излучения $\theta_{source}$ для каждого из направлений $3.26 \times 0.86$ мкрад$^2$. Размер изображения в фокусе определяется тогда как: $\sigma = f\sqrt{\theta^2_{source} +  \theta^2_{diff}}Через$, в горизонтальном направлении $42$ мкм и в вертикальном направлении $14$ мкм, что хорошо совпадает с результатами моделирования.  
\section{Интерференционный эксперимент}
Чтобы наглядно продемонстрировать эффекты, связанные с частичной когерентностью, показательно будет провести классический опыт Юнга (двухщелевой интерферометр Юнга). Ниже на Рис.~\ref{fig:double_slit_size_corr} приведён размер излучения на $25$ м от источника излучения и распределение корреляционной функции в увеличенном масштабе с наложенными щелями. Щели на рисунках обозначены разными цветами: зелёный цвет -- межщелевой зазор $75$ мкм, красный -- $150$ мкм и оранжевый -- $300$ мкм, при среднеквадратичном отклонении функции взаимной когерентности $40 \times 150$ $\textup{мкм}^2$. 
\begin{figure}[H]
	\centering
	\includegraphics[width=0.75\linewidth]{double_slit_size_corr.png}
	\caption{Размер излучения на $25$ м от источника (слева) и пятно когерентности на $25$ м от источника с щелями (справа), щели обозначены цветными полосками. Три набора щелей для каждого из направлений. Межщелевые расстояния: зелёные полоски -- $75$ мкм, красные -- $150$ мкм и оранжевые -- $300$ мкм.}
	\label{fig:double_slit_size_corr}
\end{figure}
\noindent Схема эксперимента представлена на Рис.~\ref{fig:double slit experiment}.
\begin{figure}[H] 
	\centering 	\includegraphics[width=0.99\linewidth]{double_slit_scheme.png}
	\caption{Схема двухщелевого эксперимента. После щелей -- интерферограмма усреднённа по 400 реализациям, а за ней интерферограммы для отдельных реализаций из статистического набора. Интерферограммы приведены в дальней зоне. Примечательно, что видность каждой из реализаций равна единице, но при усреднении по многим реализациям видность падает из-за наличия частичной когерентности излучения}
	\label{fig:double slit experiment}
\end{figure}
\noindent Как отмечает Дж. Гудмен, «Статистическая оптика», издательство «Мир», 1988, стр. $332-333$ или зарубежное издание~\cite{goodman_statistical_2015}:  «Хотя видность любой из этих отдельны интерферограмм соответствует значению\footnote{$\mu_{12}$ -- комплексный коэффициент когерентности, соответствующий $g^{(1)}$ в настоящей работе} $|\mu_{12}| = 1$, видность суперпозиции интерферограмм, вообще говоря, будет иной, поскольку фазы отдельных компонент будут изменяться от реализации к реализации. Таким образом, интерферограмма, усреднённая по ансамблю, вообще говоря, даст значения $|\mu_{12}|$, весьма отличные от единицы.»

Интерферограммы для различных межщелевых расстояний показаны на Рис.~\ref{fig:x_slits_75},~\ref{fig:x_slits_150},~\ref{fig:x_slits_300}. Распределения представлены для \textit{вертикального} расположения щелей.
\begin{figure}[H]
	\centering
	\begin{subfigure}{0.33\textwidth}
		\centering
		\includegraphics[width=1\linewidth]{x_slits_width_3e-05_separation_7.5e-05_.png}
		\caption{}
		\label{fig:x_slits_75}
	\end{subfigure}
	\begin{subfigure}{0.33\textwidth}
		\centering
		\includegraphics[width=1\linewidth]{x_slits_width_3e-05_separation_0.00015_.png}
		\caption{}
		\label{fig:x_slits_150}
	\end{subfigure}\hfill
	\begin{subfigure}{0.33\textwidth}
		\centering
		\includegraphics[width=1\linewidth]{x_slits_width_3e-05_separation_0.0003_.png}
		\caption{}
		\label{fig:x_slits_300}
	\end{subfigure}
	\caption{Дифракционные картины для \textit{вертикального} расположения щелей в дальней зоне, слева на право межщелевой зазор: $75$ мкм, $150$ мкм, $300$ мкм. Цвета линий соответствуют цветам на Рис.~\ref{fig:double_slit_size_corr}}
	\label{fig:x_slits}
\end{figure}
\noindent Необходимо заметить, что эти интерференционные картины представлены в $k$-пространстве или, другими словами, в дальней зоне на расстоянии $z \to \infty$ от щелей. Ещё одна особенность получившихся изображений: щели имеют конечный, в данном случае, горизонтальный размер равный $1$ мм, именно поэтому в вертикальном направлении на Рис.~\ref{fig:x_slits_75},~\ref{fig:x_slits_150},~\ref{fig:x_slits_300} видна дифракция от полуплоскости.

Аналогичные интерферограммы построены для \textit{горизонтальной} ориентации щелей (Рис.~\ref{fig:y_slits}).
\begin{figure}[H]
	\centering
	\begin{subfigure}{0.33\textwidth}
		\centering
		\includegraphics[width=1\linewidth]{y_slits_width_3e-05_separation_7.5e-05_.png}
		\caption{}
		\label{fig:y_slits_75}
	\end{subfigure}
	\begin{subfigure}{0.33\textwidth}
		\centering
		\includegraphics[width=1\linewidth]{y_slits_width_3e-05_separation_0.00015_.png}
		\caption{}
		\label{fig:y_slits_150}
	\end{subfigure}\hfill
	\begin{subfigure}{0.33\textwidth}
		\centering
		\includegraphics[width=1\linewidth]{y_slits_width_3e-05_separation_0.0003_.png}
		\caption{}
		\label{fig:y_slits_300}
	\end{subfigure}
	\caption{Дифракционные картины для \textit{горизонтального} расположения щелей в дальней зоне, слева на право межщелевой зазор: $75$ мкм, $150$ мкм, $300$ мкм. Цвета линий соответствуют цветам на Рис.~\ref{fig:double_slit_size_corr}}
	\label{fig:y_slits}
\end{figure}
\noindent В этом направлении излучение обладает большей степенью когерентности.

\section{Отражение от неидеального зеркала}\label{section:roughness}
В этом разделе мы рассмотрим применение СЕРВАЛа при расчёте отражения частично когерентного излучения от зеркал с шероховатостями. При отражении от неидеального зеркала волновой фронт деформируется, что может в значительной степени влиять на размер и максимальную интенсивность излучения в фокусе, а также на когерентные свойства излучения. Ошибки по высоте $\delta h$ вносят фазовый сдвиг: 
\begin{align}
	\phi = \cfrac{4 \pi \delta h}{\lambda} \sin{(\theta_i)},
	\label{eq:roughness}
\end{align}
где $\theta_i$ -- угол падения на зеркало, отсчитываемый от поверхности. 

Формула~\ref{eq:roughness} даёт простой путь учёта шероховатости поверхностей при моделировании в волновом подходе. Таким образом действие неидеальной поверхности учитывается как фазовый фактор, модулирующий волновой фронт. Альтернативный подход -- использование пошагового моделирования процесса отражения волнового фронта от поверхности зеркала с учётом прохождения излучения в вещество, так называемый ангд. split-step method~\cite{serkez_design_2015}. Сравнительный анализ этих двух походов приведён в работе~\cite{serkez_design_2015}, где показано совпадение оценки числа Штреля для различных величин шероховатостей. Вопросы моделирования профиля зеркала $\delta h$ рассматриваются в Приложении~\ref{AppendixB}

Для моделирования была выбрана та же оптическая схема, что и для фокусировки с апертурой в разделе~\ref{section:focusing_system_with_aperture}. В данном примере в качестве фокусирующих элементов рассматриваются зеркала с тем же фокусным расстоянием -- 12.5 метра. Апертура была исключена из рассмотрения, чтобы показать действие неидеального зеркала на свойства излучения при отражении. 

На Рис.~\ref{fig:x_SERVAL_radiaiton_after_reflection_2d_3A} и~\ref{fig:y_SERVAL_radiaiton_after_reflection_2d_3A} представлены распределения излучения после отражения от неидеальных фокусирующих зеркал в \textit{двух случаях}: вертикального расположения зеркала (Рис.~\ref{fig:x_SERVAL_radiaiton_after_reflection_2d_3A}) и горизонтального (Рис.~\ref{fig:y_SERVAL_radiaiton_after_reflection_2d_3A}). Среднеквадратичная амплитуда шероховатостей зеркала $h_{rms}$ в каждом случае составляет $0,3$ нм. Распределение излучения представлено после отражения и распространения волнового фронта на $12,5$ м через пустое пространство.
\begin{figure}[H]
	\centering
	\begin{subfigure}{0.49\textwidth}
		\centering
		\includegraphics[width=1\linewidth]{x_SERVAL_radiaiton_after_reflection_2d_3.0_A.png}
		\caption{}
		\label{fig:x_SERVAL_radiaiton_after_reflection_2d_3A}
	\end{subfigure}
	\begin{subfigure}{0.49\textwidth}
		\centering
		\includegraphics[width=1\linewidth]{y_SERVAL_radiaiton_after_reflection_2d_3.0_A.png}
		\caption{}
%{Распределение интенсивности излучения после отражения и $12,5$ м пустого пространства, ошибки по высоте введены по вертикальному направлению}
		\label{fig:y_SERVAL_radiaiton_after_reflection_2d_3A}
	\end{subfigure}
	\caption{Распределение интенсивности излучения после отражения от неидеального фокусирующего зеркал и распространение через $12,5$ метров пустого пространства. На Рис.~\ref{fig:x_SERVAL_radiaiton_after_reflection_2d_3A} зеркало ориентировано \textit{вертикально}, на Рис.~\ref{fig:y_SERVAL_radiaiton_after_reflection_2d_3A} зеркало ориентировано \textit{горизонтально}.}
	\label{fig:SERVAL_radiaiton_after_reflection_2d_3A}
\end{figure}
\noindent Видно, при вертикальном расположении зеркала модуляции волнового фронта более сглажены из-за меньшей степени когерентности в горизонтальном направлении, в отличии горизонтального расположения зеркала -- направление с большей степенью когерентности.

Для сравнения того, как влияют разные профили зеркала на свойства излучения после пропагации и в фокусе, на Рис.~\ref{fig:x_SERVAL_radiaiton_after_reflection},~\ref{fig:y_SERVAL_radiaiton_after_reflection},~\ref{fig:x_SERVAL_radiaiton_in_focus},~\ref{fig:y_SERVAL_radiaiton_in_focus} представлены соответствующие распределения интенсивности излучения после отражения от зеркала со среднеквадратичными амплитудами шероховатостей равными $0,3$ нм и $0,6$ нм. Сравнение приведено для \textit{двух ориентаций} моделируемого зеркала: зеркало ориентировано вертикально Рис.~\ref{fig:x_SERVAL_radiaiton_after_reflection},~\ref{fig:x_SERVAL_radiaiton_in_focus}. и, соответственно, горизонтально на Рис.~\ref{fig:y_SERVAL_radiaiton_after_reflection},~\ref{fig:y_SERVAL_radiaiton_in_focus}. Для всех случаев была выбрана $PSD$ функция с коэффициентом $\beta = -1,8$, нормированная на соответствующие значения среднеквадратического отклонения ошибок профиля. Характер $PSD$ функции разобран в Приложении~\ref{AppendixB}.

\begin{figure}[H] 
	\centering 	\includegraphics[width=0.99\linewidth]{x_SERVAL_radiaiton_after_reflection.png}
	\caption{Распределение интенсивности излучения после пропагции на 12.5 от зеркала, зеркало ориентировано \textit{вертикально}}
	\label{fig:x_SERVAL_radiaiton_after_reflection}
\end{figure}

\begin{figure}[H] 
	\centering 	\includegraphics[width=0.99\linewidth]{y_SERVAL_radiaiton_after_reflection.png}
	\caption{Распределение интенсивности излучения после распространение на 12.5 от зеркала, зеркало ориентировано \textit{горизонтально}}
	\label{fig:y_SERVAL_radiaiton_after_reflection}
\end{figure}
\noindent При генерации профиля зеркала выбиралось одинаковое начальное число (англ. seed) для генерации псевдослучайной величины, что делалось для воспроизводимости результатов. Именно поэтому модуляции распределение на Рис.~\ref{fig:x_SERVAL_radiaiton_after_reflection} и~\ref{fig:y_SERVAL_radiaiton_after_reflection} при увеличении величины шероховатости просто повторяют свою форму, но с большей амплитудой.

На Рис.~\ref{fig:x_SERVAL_radiaiton_in_focus} и~\ref{fig:y_SERVAL_radiaiton_in_focus} представлены распределения интенсивности излучения на образце (в фокусе).
\begin{figure}[H] 
	\centering 	\includegraphics[width=0.99\linewidth]{x_SERVAL_radiaiton_in_focus.png}
	\caption{Распределение интенсивности излучения в фокусе, зеркало в данном случае ориентировано \textit{вертикально}}
	\label{fig:x_SERVAL_radiaiton_in_focus}
\end{figure}

\begin{figure}[H] 
	\centering 	\includegraphics[width=0.99\linewidth]{y_SERVAL_radiaiton_in_focus.png}
	\caption{Распределение интенсивности излучения в фокусе, зеркало ориентировано \textit{горизонтально}}
	\label{fig:y_SERVAL_radiaiton_in_focus}
\end{figure}

Как видно на Рис.~\ref{fig:x_SERVAL_radiaiton_in_focus} и~\ref{fig:y_SERVAL_radiaiton_in_focus} шероховатости приводят к расплыванию фокусного пятна и, следственно, падению пиковой амплитуды интенсивности поля. Этот эффект играет критическую роль, например, для монохроматоров, основанных на дифракционных решётках, где размер фокусного пятна определяет разрешающую способность монохроматора. Поэтому следует налагать довольно жёсткие требования на среднеквадратичные амплитуды шероховатостей рентгеновских зеркал,~\cite{strocov_high-resolution_2010},~\cite{sankari_hippie_nodate}. 

