\chapter{Теоретический базис} \label{chapt1}
В настоящей главе будут разобраны элементы статистической оптики, а именно одна из основных теорем статистической оптики, теорема Ван Циттерта - Цернике. Используя вывод теоремы, в качестве примера, оценивается пятно когерентности излучения для рентгеновской трубки -- полностью некогерентного источника излучения, и далее даётся формулировка обобщённой теоремы Ван Циттерта - Цернике в случае, когда на источнике излучения есть конечная область когерентности, или другими словами источник частично когерентен. Так же в главе рассматриваются вопросы формирования синхротронного излучения от электронного пучка с конечным эмиттансом. Обсуждаются статистические свойства такого излучения и описывается характерная спайковая структура излучения для одной статистической реализации поля.
\section{Распространение функции взаимной когерентности и теорема Ван Циттерта - Цернике}
Распространение функции взаимной когерентности~\ref{eq:g1} поля $E(r, t)$ через свободное пространство от некогерентных стационарных источников излучения описывается теоремой Ван Циттерта - Цернике \cite{van_cittert_wahrscheinliche_1934}, \cite{zernike_concept_1938}. 
\begin{align}
	\Gamma^{(1)} (r_1, r_2, t) = \cfrac{\big \langle {E}(r_1, t) {E}(r_2, t) \big \rangle}{\big \langle {E}(r_1, t)\big \rangle \big \langle{E}(r_2, t) \big \rangle}, 
	\label{eq:g1} 
\end{align}
где $\big \langle ... \big \rangle$ означает усреднение по статистическим реализациям поля. Теорема даёт связь между распределением интенсивности источника излучения $I(\xi, \eta)$ и функцией взаимной когерентности $\Gamma^{(1)} (r_1, r_2, t)$ через двумерное Фурье преобразование. 
\begin{align}
	\Gamma^{(1)} (r_1, r_2, t) = \cfrac{\kappa e^{-i\psi}}{(\bar{\lambda}z)^2} \iint \limits_{-\infty}^{+\infty} I(\xi, \eta) \exp{\big [(i \cfrac{2 \pi}{{\lambda}z}) (\Delta x \xi + \Delta y \eta)\big]}d\xi d\eta, 
	\label{eq:van_cittert_zernike_theorem} 
\end{align}
где $\kappa = {\lambda}^2 / \pi$, $\bar{\lambda}$ -- средняя длина волны квазимонохроматического источника излучения, $z$ -- расстояние до плоскости наблюдения от источника излучения, $\psi = \cfrac{\pi}{\bar{\lambda} z}\big[((x^2_2 + y^2_2) - (x^2_1 + y^2_1)) \big]$, а $\Delta x = x_2 - x_1$, $\Delta y = y_2 - y_1$, а геометрические величины изображены на Рис.~\ref{fig:VCC_scheme_incoh} 
\begin{figure}[H] 
	\centering 	\includegraphics[width=0.99\linewidth]{VCC_incoh_scheme.png}
	\caption{К формулировке теоремы Ван Циттерта-Цернике}
	\label{fig:VCC_scheme_incoh}
\end{figure}
Таким образом площадь пятна когерентности на расстоянии $z$ от источника будет определяться следующим выражением: 
\begin{align}
	A_c = \cfrac{({\lambda} z)^2}{A_s}.
	\label{eq:VCC}
\end{align}

Теорема может быть видоизменена и обобщена для частично когерентных источников излучения достаточно лишь заменить $\kappa$ на двойной интеграл \cite{goodman_statistical_2015}
\begin{align}
	\kappa(\bar{x}, \bar{y}) = \iint \limits_{-\infty}^{+\infty} \mu(\Delta \xi, \Delta \eta) \exp{\big [(i \cfrac{2 \pi}{\bar{\lambda}z}) ( \bar{x} \Delta \xi + \bar{y} \Delta \eta)\big]}d\Delta \xi d\Delta \eta, 
\end{align}
где $\bar{x} = \cfrac{x_1 + x_2}{2}, \bar{y} = \cfrac{y_1 + y_2}{2}$,  $\Delta \xi = \xi_2 - \xi_1$, $\Delta \eta = \eta_2 - \eta_1$ и $\mu(\Delta \xi, \Delta \eta)$ -- комплексный коэффициент когерентности, по сути, область когерентности на источнике. Физически это значит следующее, Рис.~\ref{fig:VCC_scheme_partially}: огибающая излучения в дальней зоне будет обратно пропорциональна пятну когерентности излучения на источнике, а характерный размер когерентности в плоскости $xy$ на расстоянии $z$ обратно пропорционален размеру источника излучения -- интегральный множитель в формуле~\ref{eq:van_cittert_zernike_theorem}.
\begin{figure}[H] 
	\centering 	\includegraphics[width=0.99\linewidth]{VCC_partially_coh_scheme.png}
	\caption{К формулировке теоремы Ван Циттерта-Цернике}
	\label{fig:VCC_scheme_partially}
\end{figure}

В качестве примера распространения когерентности от полностью некогерентного источника можно оценить область когерентности излучения лабораторной рентгеновской трубки. Область когерентности от полностью некогерентного источника излучения квадратной формы получается напрямую из теоремы Ван Циттерта - Цернике. Подставляя в уравнение~\ref{eq:VCC} $z = 1$ м и $\lambda \approx 0.7$ $\textup{\AA}$ с площадью фокального пятна, спроецированного на направление выхода излучения из рентгеновской трубки, равной порядка $A_s = 1$ $\textup{мм}^2$, \cite{cullity_elements_1956}. Таким образом линейный размер длины когерентности при отражении от исследуемого кристалла с учётом угла дифракции ($\sim 45^{\circ}$) будет порядка $0.1$ $\textup{мкм}$. Однако линейный размер пятна когерентности может быть увеличен до нескольких микрон при использовании трубки с вращающимся анодом, где характерный диаметр круглого источника достигает $50$ $\textup{мкм}$ \cite{cullity_elements_1956}.  

Для синхротронных источников излучения область когерентности на источнике определяется натуральным размером излучения одного электрона при пролёте через вставное устройство. Например, в случае ондуляторного источника, натуральный размер излучения определяется геометрическим размером перетяжки излучения в центре ондулятора -- $\sigma_r = \sqrt{\lambda L}/4 \pi$, где $L$ длина ондулятора. Точные выражения представлены в~\cite{geloni_transverse_2008}. 

\section{О статистических свойства синхротронного излучения}
Излучение от всего электронного пучка может быть представлено как сумма полей от каждого электрона, где $k$-ый электрон в пучке имеет свою координату -- $\vec{\eta}_k$, угол -- $\vec{\l}_k$, отсчитываемые от проектной траектории, а также время прибытия $t_k$ относительно некоторого времени $t_0$. Ондуляторное излучение удобно рассматривать в $\omega$-пространстве, т.е. $\bar{E}(\vec{r}, \omega)$ связано с полем $E(r, t)$ обратным преобразование Фурье. Вклад времени прибытия в $r\omega$-пространстве будет простым умножением поля на фазовый фактор $\exp{(i \omega t_k)}$. Указанные величины $\vec{\eta}_k$, $\vec{\l}_k$ и $t_k$ подчиняются некоторым распределениям плотности вероятности, для накопительных колец в модельных случаях это распределение Гаусса. Также здесь не рассматривается разброс электронов по энергии, влияние энергетического разброса электронного пучка на когерентные свойства излучения описано в \cite{geloni_effects_2018}. Результирующее поле от $N_e$ электронов на расстоянии $z$ от источника можно записать следующим образом:
\begin{align}
	\bar{E}_{b} (z, \vec{r}, \omega) = \sum\limits_{k=1}^{N_e} \bar{E}(\vec{\eta}_k, \vec{\l}_k, z, \vec{r}, \omega) \exp{(i \omega t_k)},
	\label{eq:E_bunch} 
\end{align}
для электронов в накопительных кольцах случайные величины $\vec{\eta}_k$ и $\vec{\l}_k$ не зависят от времени прибытия $t_k$ и, в центре ондулятора, независимы друг от друга. Модуль поля $|\bar{E}_k|$ имеет одинаковое распределение для всех $k$ со средним $\big \langle|\bar{E}_k|\big \rangle$ и конечным вторым моментом  $\big \langle|\bar{E}_k|^2\big \rangle$, где $\langle \cdot \rangle$ усреднение по статистическим реализациям.

Результирующее поле $\bar{E}_{b}$ является суммой вкладов от каждого электрона в пучке и по своей структуре в правой части уравнения~\ref{eq:E_bunch} записан некоторый фазор. Следуя предпосылкам центральной предельной теоремы (ЦПТ), можно показать, что поле $\bar{E}_{b}(z, \vec{r}, \omega)$ в каждой точке $r$ подчиняется комплексному гауссовому распределению для двух практически значимых предельных случаев: случай длинного $\omega\sigma_T \gg 1$ и короткого электронного пучка $\omega\sigma_T \ll 1$, где $\sigma_T$ -- длительность электронного пучка,~\cite{geloni_transverse_2008}. В случае длинного электронного пучка величина $\omega t_k$ равномерно распределена в пределах от $0$ до $2\pi$ и продольная длина когерентности излучения на фундаментальной гармонике определятся натуральной длительность излучения от одного электрона $\lambda N /c$, где $N$ количество периодов ондулятора, которая, соответственно, мала по сравнение с длительностью электронного пучка. Этот случай будет, для определённости, называться продольно некогерентным. Для короткого электронного пучка фазовый множитель $\exp{(i \omega t_k)}$ может быть взят равным единице и излучения является продольно когерентным. В целом, формула~\ref{eq:E_bunch} даёт прямой путь моделирования синхротронного излучения с любой степенью когерентности, с учётом продольной когерентности/некогрентности излучения.

Как уже было отмечено амплитуда поля по формуле~\ref{eq:E_bunch} обладает спайковой структурой, и, что важно отметить, как в $\omega t$-пространстве, так и в поперечном направлении~\ref{fig:spikes}, В итоге, получается некая трёхмерная структура с флуктуирующей амплитудой поля, при распространение через пустое пространство, меняющая своё распределение, Рис.~\ref{fig:spikes}. 
\begin{figure}
	\centering 	\includegraphics[width=0.99\linewidth]{spikes.png}
	\caption{Спайковая структура излучения синхротронного излучения.}
	\label{fig:spikes}
\end{figure}
В $t$-пространстве поле имеет внутреннюю структуру с характерным размером спайка равному продольной длине когерентности излучения от одно электрона, а характерная длительность импульса поля, усреднённого по многим реализациям, определяется длительностью электронного пучка. Соответственно, в $\omega$-пространстве размер спайка в спектре обратно пропорционален длительности излучения, а характерная огибающая спектра, после усреднения по многим реализациям, обратно пропорциональна длине когерентности излучения, такое соотношение -- следствие теоремы Винера-Хинчина. Если разрешить монохроматором (красная линия на Рис.~\ref{fig:spikes}) спайк в $\omega$-пространстве, то на двумерном детекторе в дальней зоне можно увидеть поперечную спайковую структуру синхротронного излучения, как на Рис.~\ref{fig:spikes}. Это распределение с точностью до фазового фактора связано с распределением излучения на образце Фурье-преобразованием. В дальней зоне характерный размер спайка связан с размером источника излучения как:  $(\omega d /c)^{-1}$, и огибающая поля -- усреднённое по многим реализациям: $(\omega \Delta /c)^{-1}$, -- что является следствием обобщённой теоремы Ван Циттерт - Цернике. 


\newpage
%============================================================================================================================






