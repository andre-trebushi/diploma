\chapter{Теоретический базис} \label{chapt1}

Распространение функции взаимной когерентности~\ref{eq:g1} поля $E(r, \omega)$ через свободное пространство от некогерентных стационарных источников излучения описывается теоремой Ван Циттера - Цирнике \cite{van_cittert_wahrscheinliche_1934}, \cite{zernike_concept_1938}. 
\begin{align}
	g^{(1)} (r_1; r_2) = \cfrac{\big \langle \bar{E}(r_1) \bar{E}(r_2) \big \rangle}{\big \langle \bar{E}(r_1)\big \rangle \big \langle\bar{E}(r_2) \big \rangle}, 
	\label{eq:g1} 
\end{align}
где $\big \langle ... \big \rangle$ означает усреднение по статистическим реализациям поля. Теорема даёт связь между распределением интенсивности источника излучения $I(\xi, \eta)$ и функцией взаимной когерентности $g^{(1)} (r_1, r_2)$ через двумерное Фурье преобразование. 
\begin{align}
	g^{(1)} (x_1, y_1, x_2, y_2) = \cfrac{\kappa e^{-i\psi}}{(\bar{\lambda}z)^2} \iint \limits_{-\infty}^{+\infty} I(\xi, \eta) \exp{\big [(i \cfrac{2 \pi}{\bar{\lambda}z}) (\Delta x \xi + \Delta y \eta)\big]}d\xi d\eta, 
	\label{eq:van_cittert_zernike_theorem} 
\end{align}
где $\kappa = \bar{\lambda}^2 / \pi$, $\bar{\lambda}$ -- средняя длина волны квазимонохроматического источника излучения, $z$ -- расстояние до плоскости наблюдения от источника излучения, $\psi = \cfrac{\pi}{\bar{\lambda} z}\big[((x^2_2 + y^2_2) - (x^2_1 + y^2_1)) \big]$, а $\Delta x = x_2 - x_1$, $\Delta y = y_2 - y_1$

Теорема может быть видоизменена и сформулирована для частично когерентных источников излучения достаточно лишь заменить $\kappa$ на двойной интеграл \cite{goodman_statistical_2015}
\begin{align}
	\kappa(\bar{x}, \bar{y}) = \iint \limits_{-\infty}^{+\infty} \mu(\Delta \xi, \Delta \eta) \exp{\big [(i \cfrac{2 \pi}{\bar{\lambda}z}) ( \bar{x} \Delta \xi + \bar{y} \Delta \eta)\big]}d\Delta \xi d\Delta \eta, 
\end{align}
где $\bar{x} = \cfrac{x_1 + x_2}{2}, \bar{y} = \cfrac{y_1 + y_2}{2}$,  $\Delta \xi = \xi_2 - \xi_1$, $\Delta \eta = \eta_2 - \eta_1$. Таким образом, следую модифицированной теореме ван Циттер-Цирнике, область пятна когерентности на расстоянии $z$ будет определяться не только размером источника излучения, но и размером области когерентности на самом источнике. 

В качестве примера распространения когерентности от полностью некогерентного источника можно оценить область когерентности излучения лабораторной рентгеновской трубки. Область когерентности от полностью некогерентного источника излучения квадратной формы получается напрямую из теоремы ван Циттер-Цирнике
\begin{align}
	A_c = \cfrac{(\bar{\lambda} z)^2}{A_s}.
\end{align}
Подставляя $z = 1$ м и $\lambda \approx 0.7$ $\textup{\AA}$ со спроецированной на направление выхода излучения из рентгеновской трубки площадью фокального пятна меньше чем $A_s = 1$ $\textup{мм}^2$ \cite{cullity_elements_1956}. Таким образом линейный размер длины когерентности при отражении от исследуемого кристалла с учётом угла дифракции ($= 45^{\circ}$) будет порядка $0.1$ $\textup{мкм}$. Однако линейный размер пятна когерентности может быть увеличен до нескольких микрон при использовании трубки с вращающимся анодом, где характерный диаметр круглого источника достигает $50$ $\textup{мкм}$ \cite{cullity_elements_1956}.  

Для оценки когерентных свойств источников синхротронного излучения обязательно рассмотрение области когерентности на самом источнике так как именно область когерентности на источнике будет в значительной степени формировать когерентные свойства излучения в дельней зоне -- на расстоянии $z$ от источника. Для синхротронных источников излучения источником излучения является релятивистский электронный пучок однако для оценки размера когерентности для начала следует рассмотреть излучение одного электрона при пролёте через магнитную систему. В качестве магнитной системы будет рассматриваться ондуляторное излучение на фундаментальной гармонике. Ондуляторное излучение имеет мнимый источник излучения в центре ондулятора, который обладает плоским фазовым фронтом, подобно гауссову пучку в центре лазерного резонатора, этим ондуляторное излучение схоже с лазерным. Поля в центре ондулятора даётся выражением \cite{geloni_fourier_2007}:
\begin{align}
	\widetilde{E}_{\bot}(0, \vec{r}_{\bot}) =
	i \frac{e A_{JJ} \omega}{2 c^2}\frac{K}{\gamma} \times \bigg [\pi - 2\text{Si} \bigg( \cfrac{i \omega \vec{r}_{\bot}^{2}}{L_w c}\bigg)\bigg].
	\label{eq:single_electron_near_field_z=0}
\end{align}
Линейный размер мнимого источника излучения или размер перетяжки $\sigma_{r} = \cfrac{\sqrt{\lambda L_w}}{4 \pi}$ в источнике \cite{geloni_brightness_2014} определят пятно когерентности на источнике \cite{geloni_transverse_2008}. Другими словами, ондуляторное излучение от одного электрона полностью поперечно когерентно. Размер зоны когерентности на источнике задаёт нормированный размер электронного пучка $N_{x, y} = \cfrac{2 \pi \sigma^2_{x, y}}{\lambda L_w}$, где $\sigma^2_{x, y}$ --- $x-$ и $y-$ линейные размеры электронного пучка, в то же время длина волны излучения $\lambda$ задаёт масштаб, объекты меньшие длины волны вообще не различаются. Если размер электронного пучка много меньше, то весь электронный пучок будет излучать как один электрон -- когерентно, если размер электронного пучка больше размер пятна когерентности на источнике, то когерентно будут излучать только части электронного пучка.

\rr{привести практический пример из работы GG}
\newpage
%============================================================================================================================
