\chapter*{Заключение}						% Заголовок
%\addcontentsline{toc}{chapter}{Заключение}	% Добавляем его в оглавление

%% Согласно ГОСТ Р 7.0.11-2011:
%% 5.3.3 В заключении диссертации излагают итоги выполненного исследования, рекомендации, перспективы дальнейшей разработки темы.
%% 9.2.3 В заключении автореферата диссертации излагают итоги данного исследования, рекомендации и перспективы дальнейшей разработки темы.
%% Поэтому имеет смысл сделать эту часть общей и загрузить из одного файла в автореферат и в диссертацию:

В работе представлен новый метод моделирования частично когерентного синхротронного излучения, называемый СЕРВАЛ. Метод основан на ограничении пространственных гармоник комплексного гауссового шума эффективным размером и расходимостью излучения на источнике. СЕРВАЛ даёт точную оценку распределения поля частично когерентного синхротронного излучения и поперечной функции когерентности. В работе проведён сравнительный анализ СЕРВАЛа и метода сложения амплитуд, показано, что СЕРВАЛ имеет преимущество в быстродействии на два порядка. В работе представлено применение СЕРВАЛа для расчёта оптических схем. Рассмотрено отражение частично когерентного излучения от рентгеновского зеркала с шероховатостями. 

СЕРВАЛ нашёл применение при проектировании рентгенооптических трактов источника синхротронного излучения четвёртого поколения ЦКП «СКИФ». С помощью СЕРВАЛа было смоделировано прохождение частично когерентного синхротронного излучения через толстый образец с градиентом диэлектрической проницаемости. Решаемая задача представляет первое приближение численной модели рассеяния рентгеновского излучения на фронте ударной волны. 
\newpage
%%% Согласно ГОСТ Р 7.0.11-2011:
%% 5.3.3 В заключении диссертации излагают итоги выполненного исследования, рекомендации, перспективы дальнейшей разработки темы.
%% 9.2.3 В заключении автореферата диссертации излагают итоги данного исследования, рекомендации и перспективы дальнейшей разработки темы.
\begin{enumerate}
  \item Были выработаны оптимальные критерии отбора событий процесса ${e^+\:e^- \to K_{S}\:K_{L}\:\pi^0}$. Вклад данного процесса в общее адронное сечение менее 5\%, тем не менее удалось набрать статистику в 1924 событий при помощи выбора оптимальных условий отбора.
  \item Определена эффективность регистрации при помощи Монте-Карло моделирования, которая составила около 2\% в диапозоне энергий от \text{1.1 ГэВ} до \text{1.4 ГэВ} и около 4.5\% для энергий от 1.4 ГэВ до 2 ГэВ со статистической ошибкой $\sim1\%$.
  \item Измерено сечение процесса ${e^+\:e^- \to K_{S}\:K_{L}\:\pi^0}$ в диапозоне энергий от \text{1.1 ГэВ} до \text{2 ГэВ} со статистической погрешностью около 10\%. Вычислены радиационные поправки с точностью $\sim1\%$, с последующим получением конечного результата --- борновское сечение изучаемого процесса.
  \item Сделана оценка вклада трех основных фоновых событий, общий вклад которых составил не более 10\% от сечения процесса ${e^+\:e^- \to K_{S}\:K_{L}\:\pi^0}$.
\end{enumerate}

\chapter*{Благодарности}						% Заголовок