\chapter*{Заключение}						% Заголовок
%\addcontentsline{toc}{chapter}{Заключение}	% Добавляем его в оглавление

%% Согласно ГОСТ Р 7.0.11-2011:
%% 5.3.3 В заключении диссертации излагают итоги выполненного исследования, рекомендации, перспективы дальнейшей разработки темы.
%% 9.2.3 В заключении автореферата диссертации излагают итоги данного исследования, рекомендации и перспективы дальнейшей разработки темы.
%% Поэтому имеет смысл сделать эту часть общей и загрузить из одного файла в автореферат и в диссертацию:

Основные результаты работы заключаются в следующем.
%%% Согласно ГОСТ Р 7.0.11-2011:
%% 5.3.3 В заключении диссертации излагают итоги выполненного исследования, рекомендации, перспективы дальнейшей разработки темы.
%% 9.2.3 В заключении автореферата диссертации излагают итоги данного исследования, рекомендации и перспективы дальнейшей разработки темы.
\begin{enumerate}
  \item Были выработаны оптимальные критерии отбора событий процесса ${e^+\:e^- \to K_{S}\:K_{L}\:\pi^0}$. Вклад данного процесса в общее адронное сечение менее 5\%, тем не менее удалось набрать статистику в 1924 событий при помощи выбора оптимальных условий отбора.
  \item Определена эффективность регистрации при помощи Монте-Карло моделирования, которая составила около 2\% в диапозоне энергий от \text{1.1 ГэВ} до \text{1.4 ГэВ} и около 4.5\% для энергий от 1.4 ГэВ до 2 ГэВ со статистической ошибкой $\sim1\%$.
  \item Измерено сечение процесса ${e^+\:e^- \to K_{S}\:K_{L}\:\pi^0}$ в диапозоне энергий от \text{1.1 ГэВ} до \text{2 ГэВ} со статистической погрешностью около 10\%. Вычислены радиационные поправки с точностью $\sim1\%$, с последующим получением конечного результата --- борновское сечение изучаемого процесса.
  \item Сделана оценка вклада трех основных фоновых событий, общий вклад которых составил не более 10\% от сечения процесса ${e^+\:e^- \to K_{S}\:K_{L}\:\pi^0}$.
\end{enumerate}

Важно отметить, что оба метода: СЕРВАЛ и МСА, основываются на усреднение по статистическим реализациям, такой подход является новым при моделировании когерентных свойств синхротронного излучения и именно в этом состоит его основное различие от МСИ. 