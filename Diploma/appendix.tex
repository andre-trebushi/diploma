\appendix
%% Правка оформления ссылок на приложения:
%http://tex.stackexchange.com/questions/56839/chaptername-is-used-even-for-appendix-chapters-in-toc
%http://tex.stackexchange.com/questions/59349/table-of-contents-with-chapter-and-appendix
%% требует двойной компиляции
\addtocontents{toc}{\def\protect\cftchappresnum{\appendixname{} }%
\setlength{\cftchapnumwidth}{\widthof{\cftchapfont\appendixname~Ш\cftchapaftersnum}}%
}
%% Оформление заголовков приложений ближе к ГОСТ:
\sectionformat{\chapter}[display]{% Параметры заголовков разделов в тексте
    label=\chaptertitlename\ \thechapter,% (ГОСТ Р 2.105, 4.3.6)
    labelsep=20pt,
}
\renewcommand\thechapter{\Asbuk{chapter}} % Чтобы приложения русскими буквами нумеровались
\chapter{Описание шероховатости поверхностей зеркал} \label{AppendixB}
Число Штреля, упомянутое части~\ref{section:roughness}, описывает относительное падение интенсивности излучения на оси из-за наличия в оптической системе аберраций и может быть использовано для выдвижения критерия (критерий Марешаля), налагающего ограничение на среднеквадратичное отклонение $h_{rms}$, а также среднеквадратичную ошибку наклона $\mu_{rms}$,~\cite{church_specification_1993}:
\begin{align}
	\cfrac{I(0)}{I_0(0)} = 1 - 8\cfrac{\mu^2_{rms}}{\theta_i} - \bigg(\cfrac{4 \pi}{\lambda} \sin{(\theta_i)} \bigg)^2 h^2_{rms} < 0,8,
	\label{eq:Marechal_criterion}
\end{align}
здесь $I(0)$ и $I_0(0)$ интенсивности излучения на оси оптической системы в фокусе в случае оптики с аберрациями и без, соответственно. 

Как видно, формула~\ref{eq:Marechal_criterion} зависит от двух параметров $h_{rms}$ и $\mu_{rms}$. Полное описание поверхности зеркала возможно в обратном пространстве функцией спектральной плотности $PSD(k)$, которая определяется следующим образом: 
\begin{align}
	PSD(k) =  \cfrac{1}{L} \bigg| \int_{-L/2}^{L/2} \delta h(z) \exp{(-ikz)dz}  \bigg|^2
	\label{eq:PSD_def}
\end{align}
Профиль зеркала имеет фрактальную структуру~\cite{angeisky_fractal_nodate} с параметром $D$, удовлетворяющий условию $1 < D < 2$, что даёт линейный вид $PSD$-функции в лог-лог масштабе. Функция задаётся двумя параметрами: свободным членом $\alpha$ и наклоном $\beta$, при $D = (5 - \alpha)/2$. Соответственно, 
\begin{align}
	PSD_{лог}(k) =  \beta k - \alpha.
\end{align}
Через $PSD$-функцию удобно определять среднеквадратичные значения ошибок по высоте и наклону через интегралы по $PSD$ функции:
\begin{align}
	\mu^2_{rms} =  (2 \pi)^2 \int^{1/W}_{1/L} k^2 PSD(k) dk,
\end{align}

\begin{align}
	h^2_{rms} =  \int^{1/\lambda}_{1/W} PSD(k) dk,
\end{align}
где $W$ длина когерентности излучения на зеркале. Видно, что при некоторых значениях $W$ и параметрах оптической системы, величина $\mu_{rms}$ теряет свой смысл и равна нулю. 
%Например, для дифракционно ограниченных систем, когда $\theta_{diff} = 1.22 \lambda / D$ много больше чем видимый размер источника $\theta_{source} = \sigma_{x, y}/z_0$, где $z_0$ расстояние от источника до плоскости наблюдения. В этом случае всегда выполняется $W \sim \sqrt{2} L$, \cite{pardini_effect_2015}, и тогда применима формула $h_{rms} < \lambda / 4 \sqrt{5} \pi \theta_i$, что совпадает с аналогичными формулами у~\cite{serkez_design_2015} и~\cite{heimann_linac_2011}\footnote{$4 \pi \sqrt{5}  \approx 28$ }.

$h_{rms}$ и $\mu_{rms}$ -- некоторые средние, поэтому через них сложно полно описать профиль зеркала. При расчёте оптики необходимо знать саму $PSD$ функцию. Обращая формулу~\ref{eq:PSD_def}, можно моделировать профиль зеркала~\cite{hua_using_2013},~\cite{xu_statistical_2012},~\cite{barty_predicting_2009}.
\begin{align}
	\delta h(z) = \cfrac{M}{L} F^{-1} \bigg\{\sqrt{L \cdot PSD(k)} \exp{[i \psi(k)]} \bigg\}(z),
\end{align}
где $F^{-1}\{\cdot\}(z)$ обратное преобразование фурье, а $M$ -- количество точек вдоль оси $z$, $\psi(k)$ -- случайно сгенерированная фаза, удовлетворяющая условию $-\pi < \psi(k) <\pi$, равномерно распределенная в этом промежутке. Таким образом, зная профиль зеркала $\delta h(z)$ и используя формулу~\ref{eq:roughness}, можно провести моделирование процесса отражения волнового фронта от неидеального зеркала.
\normalsize% возвращаем шрифт к нормальному
